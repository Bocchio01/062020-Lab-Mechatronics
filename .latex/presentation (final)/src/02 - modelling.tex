\section{Modelling}

\begin{frame}{Scheme of MLS system}

    Even if MLS is composed of two coils, for the aim of this work we have chosen to focus on the \textbf{single coil configuration}, considering only the upper one.

    \begin{figure}[H]

        \begin{minipage}{0.40\textwidth}

            \centering

            \begin{tikzpicture}[european voltages]

                \def\radius{0.3}

                % Upper circuit
                \draw (-3, 3.5) node [right] {$+$}
                to [short] ++(0, -1)
                to [R, l^=$R$, resistors/zigs=6] ++(2, 0)
                to [variable cute inductor, i>^=$\dot{q}$, l=$L$] ++(2, 0)
                to [short] ++(0, +1) node [right] {$-$};

                % Reference system
                \draw[|->] (-1.0, +2.5) -- ++(0, -2.5 + \radius) node[left] {$z$, $\dot{z}$, $\ddot{z}$};

                % Ball
                \filldraw[fill=gray, draw=black] (0, 0) circle (\radius);

                % Upward forces
                \draw[thick, ->] (-0.1, +\radius) -- ++(0, +1.5) node[right] {$F_{\text{em}}$};
                \draw[thick, ->] (+0.0, +\radius) -- ++(0, +1.0) node[right] {$F_{\text{in}}$};
                \draw[thick, ->] (+0.1, +\radius) -- ++(0, +0.5) node[right] {$F_{\text{d}}$};

                % Downward forces
                \draw[thick, ->] (+0.0, -\radius) -- ++(0, -0.5) node[right] {$F_{\text{g}}$};

            \end{tikzpicture}

        \end{minipage}
        %
        \hfill
        %
        \begin{minipage}{0.55\textwidth}

            \begin{table}
                \centering

                \begin{tabular}{|c|l|}
                    \hline
                    \textbf{Name}   & \textbf{Description}  \\
                    \hline
                    $F_{\text{g}}$  & Gravitational force   \\
                    $F_{\text{in}}$ & Inertial force        \\
                    $F_{\text{d}}$  & Drag force            \\
                    $F_{\text{em}}$ & Electromagnetic force \\
                    \hline
                \end{tabular}

            \end{table}

        \end{minipage}

        \label{fig:MLS_scheme}

    \end{figure}

\end{frame}



\begin{frame}{Lagrange equation}

    To derive the \textbf{equations of motion}, we started from the Lagrange equation of the system:

    \begin{equation}
        \frac{d}{dt} \left( \frac{\partial \mathcal{T}}{\partial \dot{\mathbf{u}}} \right) - \frac{\partial \mathcal{T}}{\partial \mathbf{u}} + \frac{\partial \mathcal{D}}{\partial \dot{\mathbf{u}}} + \frac{\partial \mathcal{U}}{\partial \mathbf{u}} = \mathcal{Q}
        \text{, where }
        \mathbf{u} = \begin{bmatrix} z \\ q \end{bmatrix}
        \label{eq:lagrange_equation}
    \end{equation}

    The energy terms are defined as follows:

    \begin{equation}
        \begin{aligned}
            \mathcal{T} & = \frac{1}{2} m \dot{z}^2 + \frac{1}{2} L(z, \dot{q}) \dot{q}^2                                                       \\
            \mathcal{D} & = \int_{\dot{z}(\cdot)} \frac{1}{2} C_d A \rho \dot{z}^2 d\dot{z} + \int_{\dot{q}(\cdot)} R(\dot{q}) \dot{q} d\dot{q} \\
            \mathcal{U} & = -m g z - q V                                                                                                        \\
            \mathcal{Q} & = 0
        \end{aligned}
    \end{equation}

\end{frame}



\begin{frame}{Electrical components model}

    Based on experimental data, we have proposed a model for both the resistance and the coil inductance:

    \begin{equation}
        \begin{aligned}
            R & = R(I) = R_{0}                                                          \\
            L & = L(z, I) = L_{0} + L_{z} e^{-a_{z} z} + L_{I} \arctan(a_{I} I - b_{I})
        \end{aligned}
        \label{eq:model_for_inductance}
    \end{equation}

    The first and second derivatives with respect to ball position and current are:

    \begin{equation}
        \begin{aligned}
            \frac{\partial L}{\partial z}     & = -a_{z} L_{z} e^{-a_{z} z}  \\
            \frac{\partial^2 L}{\partial z^2} & = a_{z}^2 L_{z} e^{-a_{z} z}
        \end{aligned}
        \qquad
        \begin{aligned}
            \frac{\partial L}{\partial I}     & = \frac{L_{I} a_{I}}{1 + (a_{I} I - b_{I})^2}                            \\
            \frac{\partial^2 L}{\partial I^2} & = -2 \frac{L_{I} a_{I}^2 (a_{I} I - b_{I})}{(1 + (a_{I} I - b_{I})^2)^2}
        \end{aligned}
        \label{eq:model_for_inductance_derivatives}
    \end{equation}

\end{frame}



\begin{frame}{Model approximations}

    In order to simplify the model, some approximations have been considered:

    \begin{equation}
        \begin{cases}
            \frac{\partial L}{\partial I}     & \approx 0 \\
            \frac{\partial^2 L}{\partial I^2} & \approx 0 \\
            \dot{z}                           & \approx 0
        \end{cases}
        \label{eq:model_reduction_conditions}
    \end{equation}

    Notice that, from successive analysis, these approximations have proven (at least in the range of interest) to be valid.

\end{frame}



\begin{frame}{Equations of motion}

    By applying approximations of Equation (\ref{eq:model_reduction_conditions}) to the Lagrange equation (\ref{eq:lagrange_equation}), we have obtained the following equations of motion:

    \begin{equation}
        \begin{cases}
            \dot{z} = v                                                                         \\
            \dot{v} = m^{-1} \left(\frac{1}{2} \frac{\partial L}{\partial z} I^2 + m g  \right) \\
            \dot{I} = L^{-1} \left(- R I + V \right)
        \end{cases}
        \label{eq:equations_of_motion_single_coil}
    \end{equation}

    Model non-linearities are hidden in all the terms relative to the inductance $L$ and its derivatives.

    \vspace{9pt}

    Despite the applied approximations, the set of Equations (\ref{eq:equations_of_motion_single_coil}) is still \textbf{able to capture the main dynamics of the system}.

\end{frame}
