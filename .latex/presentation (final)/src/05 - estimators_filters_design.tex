\section{Estimators and Filters Design}
\label{sec:filters_estimators_design}

\begin{frame}{Overview}

    Many control strategies require the knowledge of all the system states.
    For this reason, some state estimators have been designed together with some filters to reduce the noise in the measurements.

    \begin{itemize}
        \item \textbf{Low-pass filter};
        \item \textbf{Luenberger observer}.
        \item \textbf{Kalman filter}.
        \item \textbf{Extended Kalman filter}.
    \end{itemize}

\end{frame}

\begin{frame}{Low-pass filter}

    A first order low-pass filter has been implemented:

    \begin{equation}
        G(s) = \frac{1}{\tau s + 1}
    \end{equation}

    where the time constant of the filter is:

    \begin{equation}
        \tau = \frac{1}{\omega_c}=\frac{1}{200} = 5 ms
    \end{equation}

    where $\omega_c = 10 \omega_n \approx 200 rad/s$ is  the cut-off frequency, and $\omega_n = \frac{2\pi}{T_p} \approx 20 rad/s$ is the bandwidth of the system.
    Filter delay can also be computed as:

    \begin{equation}
        \phi = -artctan(\omega_c \tau) = -5.7^{\circ}
    \end{equation}

\end{frame}



\begin{frame}{Luenberger Observer}

    In order to estimate the state of the system\footnotemark[1], the Luenberger Observer has been implemented considering by chance the following eigenvalues:

    \begin{equation}
        eig(A-LC) =
        \begin{bmatrix}
            -500 \\
            -400 \\
            -400
        \end{bmatrix}
        \label{eq:luenberger_observer_poles}
    \end{equation}

    Leading to the following observer gain:

    \begin{equation}
        L =
        \begin{bmatrix}
            900.00    & 0      \\
            201555.32 & -20.63 \\
            0         & 364.44
        \end{bmatrix}
        \label{eq:L_luemberg}
    \end{equation}

    \footnotetext[1]{For all the estimators proposed, we consider to have both $z$ and $I$ as measurements from the system and estimate $v$.}

\end{frame}



\begin{frame}{Kalman filter}

    Another estimation of the state of the system can be made with the Kalman filter, where process and noise covariance matrices are given by:

    \begin{equation}
        Q = \begin{bmatrix}
            0 & 0  & 0  \\
            0 & 10 & 0  \\
            0 & 0  & 10
        \end{bmatrix}
        \quad
        R = \begin{bmatrix}
            7.21 \cdot 10^{-10} & 0                  \\
            0                   & 4.21 \cdot 10^{-5}
        \end{bmatrix}
    \end{equation}

    The Kalman gain $K$ is then obtained:

    \begin{equation}
        K = \begin{bmatrix}
            487    & -0  \\
            119069 & -15 \\
            -911   & 453
        \end{bmatrix}
        \label{eq:kalman_gain_matrix}
    \end{equation}

    the poles of the observer are given by:

    \begin{equation}
        eig(A - K C) =
        \begin{bmatrix}
            -243.94 + 240.82i \\
            -243.94 - 240.82i \\
            -488.87 + 0i
        \end{bmatrix}
        \label{eq:K_kalman}
    \end{equation}

\end{frame}



\begin{frame}{Extended Kalman filter}

    The extended Kalman filter does a recursive state estimation for our nonlinear systems.

    \vspace{9pt}

    The design parameters are the same as the Kalman filter with the key difference that the gain matrix $K_k$ is computed online at each time step $t_k$ based on the current state estimate $\hat{x}_k$.

\end{frame}