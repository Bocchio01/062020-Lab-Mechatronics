\subsection{Kalman Filter}
\label{sec:kalman_filter}

The Kalman Filter is a powerful algorithm used for estimating the state of a dynamic system from a series of noisy measurements.
It is widely used in control systems, robotics, signal processing, and navigation due to its ability to provide real-time, optimal state estimates by considering system dynamics and measurement noise.
One of the key strengths of the Kalman Filter is the ability to provide smooth state estimates without introducing delays.

Mathematically, the filter assumes a linear system model of the form:

\begin{equation}
    \begin{aligned}
        \dot{x} & = A x + B u + G w \\
        y       & = C x + D u + v
    \end{aligned}
\end{equation}

Where $x$ is the state vector, $u$ is the control input, $y$ is the measurement, $A$, $B$, $C$ and $D$ are the system matrices, $G$ is the input noise matrix, and $w$ and $v$ are process and white measurement noise, respectively.
Among the assumptions of the Kalman filter, both the process and measurement noise are assumed to be zero-mean Gaussian white noise with known covariance $Q$ and $R$.

Supposing that the system is observable, coviances $Q$ and $R$ are known, one can compute the Kalman gain as:

\begin{equation}
    L = P C^T (C P C^T + R)^{-1}
    \label{eq:kalman_gain}
\end{equation}

Where $P$ is the solution to the algebraic Riccati equation:

\begin{equation}
    A P + P A^T - P C^T (C P C^T + R)^{-1} C P + Q = 0
\end{equation}

The Kalman gain is then used to correct the predicted state estimate based on the measurement.
The corrected state estimate is given by:

\begin{equation}
    \begin{cases}
        \dot{\hat{x}} = A \hat{x} + B u + K(y - C \hat{x}) \\
        \hat{y} = C \hat{x}
    \end{cases}
\end{equation}

\paragraph{Design}

As we have already seen in the Luenberger observer design, the poles of the observer are given by the eigenvalues of the matrix $A - KC$, and the observer is stable if the poles are placed in the left half plane of the complex plane.
The Kalman gain $K$ can be computed using Equation \ref{eq:kalman_gain}, considering as $Q$ and $R$ the following matrices:

\begin{equation}
    Q = \begin{bmatrix}
        10 & 0  \\
        0  & 10
    \end{bmatrix}
    \quad
    R = \begin{matrix}
        7.21 \cdot 10^{-10} & 0                  \\
        0                   & 4.21 \cdot 10^{-5}
    \end{matrix}
\end{equation}

By doing so, we obtain the following $K$ matrix:

\begin{equation}
    K \approx \begin{bmatrix}
        487    & -0  \\
        119069 & -15 \\
        -911   & 453
    \end{bmatrix}
    \label{eq:kalman_gain_matrix}
\end{equation}

The eigenvalues of the matrix $A - KC$ are given by:

\begin{equation}
    eig(A - K C) =
    \begin{bmatrix}
        -243.94 + 240.82i \\
        -243.94 - 240.82i \\
        -488.87 + 0i
    \end{bmatrix}
    \label{eq:K_kalman}
\end{equation}


