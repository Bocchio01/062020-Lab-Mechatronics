\section{Controllers Design}
\label{sec:controllers_design}

In this section, we move onto the design of the controllers that will be used to control the system.

As we have clarified in the previous modelling section (Section \ref{sec:modelling}), the system is highly nonlinear with respect to both position and current, and we control it by acting on the input PWM signal.

In the following, we will present three main families of controllers that have been adopted for the control of the system:

\begin{itemize}
    \item \textbf{PID Controllers}: a simple controller that uses the error signal, its history and derivative to compute the control signal (Section \ref{subsec:pid_controllers})
    \item \textbf{LQR Controllers}: a controller that minimizes a quadratic cost function to compute the control signal (Section \ref{subsec:lq_controllers})
    \item \textbf{MPC Controllers}: a controller that predicts the future evolution of the system and computes the control signal by minimizing a cost function (Section \ref{subsec:mpc_controllers})
\end{itemize}

For each of these controllers, we will briefly present their theoretical background, the design choices that have been made and assess their stability by means of Bode diagrams and Root Locus plots (when possible), or by means of eigenvalues analysis.
Experimental step responses (ranging from $10 [mm]$ to $12 [mm]$) will also be shown as a proof of controller stability in the nearby of the linearization point.

Notice that both stability and step responses are evaluated considering the linearized model at a distance of $10 [mm]$ from the upper coil.

Results and comparisons between the different controllers will be presented in the next section (Section \ref{sec:results}).

\subsection{PID Controllers}
\label{subsec:pid_controllers}

The Proportional-Integral-Derivative (PID) controller is a simple controller that uses the error signal, its history and derivative to compute the control signal.
It is a widely used controller in industry due to its simplicity and effectiveness in many applications.

The PID controller is defined by the following equation:

\begin{equation}
    u(t) = K_p e(t) + K_i \int_{0}^{t} e(\tau)dt + K_d \frac{de(t)}{dt} = K_p \left(e(t) + \frac{1}{T_i} \int_{0}^{t} e(\tau)dt + T_d \frac{de(t)}{dt}\right)
\end{equation}

Where $K_p$, $K_i$ and $K_d$ are the proportional, integral and derivative gains, respectively, $e(t)$ is the error signal, and $T_i$ and $T_d$ instead are the integral and derivative time constants, respectively.



\subsubsection{PID classical}
\label{subsubsec:pid_classic}

In its simplest form, the PID is a linear controller whose three gains are tuned based on the linearization of the system.
The controller gains are briefly described as follows: the proportional term $K_p$ provides an output proportional to the current error $e(t)$ and it helps to reduce it; the integral contribution $K_i$ accumulates the error over time to address any residual offset (steady-state error) that the proportional term cannot eliminate, and eventually it ensures the system to reach the set-point; finally, the derivative $K_d$ reacts to the rate of change of the error, predicting future behavior and adding damping to the system, and eventually it reduces overshoot and improves stability by anticipating changes.

\paragraph{Design}

Several gain parameters have been tested to find the optimal behavior for the considered system.
A first estimate  has been made observing the Bode diagram, whereas a better approximation of the parameters has been obtained using the Root Locus.
$T_i$ and $T_d$ were kept constant while changing $K_p$.
The gain parameters used to build the transfer function are reported below:

\begin{equation}
    K_p = -150 \quad K_i = -450 \quad K_d = -6.82
\end{equation}

\paragraph{Bode Diagram}

The final plots are presented in Figure \ref{fig:pid_classical_bode}.
Compared to Figure \ref{fig:bode_plot}, improvements on the behavior can be observed due to the application of the PID controller which tends to stabilize the system.

\begin{figure}[H]
    \centering
    \includegraphics[width=1\linewidth]{./img/MATLAB/controllers/PID_classical.pdf}
    \caption{Bode Plot, Root Locus and Step Response (PID classic)}
    \label{fig:pid_classical_bode}
\end{figure}

Eigenvalues of the system matrix have been computed since eigenvalues with negative real parts indicate stability, as well as a positive phase margin.
Indeed, the resulting system is overall stable.
Nevertheless, the experimental results obtained from the physical tests were not as expected.
A potential explanation could be that the classical PID may introduce some issues due to the integral path and the non-linearity of the system.
We have thus considered two expansions of the classical PID that bring improvements on the control of the system, that are the anti-windup (Section \ref{subsubsec:pid_anti_windup}) and the gain scheduling (Section \ref{subsubsec:pid_gain_scheduling}).



\subsubsection{PID with Anti-Windup correction}
\label{subsubsec:pid_anti_windup}

The Anti-windup variation of the PID is introduced in order to avoid the windup of the integration path when the saturation of the actuator occurs.
The integrator windup occurs when the actuator saturates and the integration part makes the error signal to increase.
This causes the degradation of the rise time of the step response, and possibly leading to higher overshoot.

The basic idea to avoid these issues is to apply a conditional integration.
The controller output is thus compared with the limits, and whenever there is some indication that saturation causes error accumulation, the integrator in PID controller is turned off.

\paragraph{Step Response}

Figure \ref{fig:pid_anti_windup_step_response} shows the response of the system state to a reference step input.
The stability of the dynamics can be observed, specifically considering the most relevant parameters such as the position of the sphere and the current flowing through the coils.
The analytical procedure is the same as for the classical PID, and thus the controller gains that have been used are the ones described in Section \ref{subsubsec:pid_classic}.

\begin{figure}[H]
    \centering
    \includegraphics[width=1\linewidth]{./img/MATLAB/results/step_PID_anti_windup_KF.pdf}
    \caption{Step Response (PID anti-windup)}
    \label{fig:pid_anti_windup_step_response}
\end{figure}



\subsubsection{PID with gain scheduling}
\label{subsubsec:pid_gain_scheduling}

Gain scheduling is usually used for highly non-linear systems due to the ease of the implementation and its affordability.
This method tunes PID controllers for a series of steady-state operating points of the plant.

In the considered system, the space interval where the sphere moves has been divided into several points that represent our steady-state operating conditions, and the state-space system has been linearized at each operating condition.
The set of operating conditions has to be large enough in order to get good performance everywhere, as well as the structure and the stability of the model changes when the sphere moves within the range of positions.
As a second step, the controller gains have been tuned for each of these operating points.
The controller develops a set of curves that gradually change the gain parameters from one operating position to another.
In this way the sphere can move within the overall space range.

\paragraph{Bode Diagram}

Several curves describing the system behavior corresponding to each operating point have been plotted in order to discuss the stability conditions.
Table \ref{tab:pid_gain_scheduling_gains} reports the gain parameters for each of the selected operating points.

\begin{table}[H]
    \centering

    \begin{tabular}{|c|c|c|c|}
        \hline
        $z [mm]$ & $K_p$  & $K_i$   & $K_d$   \\
        \hline
        $5$      & $-102$ & $-306$  & $-4.64$ \\
        $8$      & $-136$ & $-408$  & $-6.18$ \\
        $12$     & $-183$ & $-550$  & $-8.34$ \\
        $16$     & $-250$ & $-750$  & $-11.4$ \\
        $20$     & $-342$ & $-1030$ & $-15.5$ \\
        \hline
    \end{tabular}

    \caption{PID controller gains}
    \label{tab:pid_gain_scheduling_gains}

\end{table}

\begin{figure}[H]
    \centering
    \includegraphics[width=1\linewidth]{./img/MATLAB/controllers/PID_gain_scheduling.pdf}
    \caption{Bode plot, Root Locus and Step Response (PID gain scheduling)}
    \label{fig:pid_gain_scheduling_bode_diagram}
\end{figure}

\paragraph{Step Response}

The efficiency of the response of the system state to a reference step input is described in Figure \ref{fig:pid_gain_scheduling_step_response}.

\begin{figure}[H]
    \centering
    \includegraphics[width=1\linewidth]{./img/MATLAB/results/step_PID_gain_scheduling_KF.pdf}
    \caption{Step Response (PID gain scheduling)}
    \label{fig:pid_gain_scheduling_step_response}
\end{figure}


\subsection{LQ Controllers}
\label{subsec:lq_controllers}
Linear Quadratic (LQ) Controllers are optimal controllers that use state-space representation. These kinds of models minimize a quadratic cost function that balances state performance and control effort, providing a systematic way to design efficient and stable controllers.

Hereafter the Linear Quadratic Regulator (LQR), its expansion with tracking and Linear Quadratic Integrator (LQI) are taken into account to develop a stable controller for our system.

\subsubsection{LQR}
\label{subsubsec:lqr}
The Linear Quadratic Regulator (LQR) is a full state feedback controller. In order to provide the optimal control to the system, the controller aims to minimize the cost function $\mathcal{J}$ (Equation \ref{cost function}). The feedback control gain matrix $\mathbf{K}$ is thus computed considering the closed-loop characteristics that are relevant to us, specifically how efficient must be the and how much effort can be spent to get the desired performance.

The cost function which we aim to minimize is $\mathcal{J}$:

\begin{equation}
    \mathcal{J} = \int_0^\infty \mathbf{x}(t)^\top \mathbf{Q} \mathbf{x}(t) + \mathbf{u}(t)^\top \mathbf{R} \mathbf{u}(t) dt,
    \label{cost function}
\end{equation}

where $\mathbf{Q}$ is a positive semi-definite matrix penalizing state deviations from the desired state, and $\mathbf{R}$ is a positive semi-definite matrix penalizing control effort.

The optimal control input is given by:
\begin{equation}
    \mathbf{u}(t) = -\mathbf{K} \mathbf{x}(t),
\end{equation}

where the feedback gain $\mathbf{K}$ is determined by solving the Algebraic Riccati Equation:
\begin{equation}
    \mathbf{A}^\top \mathbf{P} + \mathbf{P}\mathbf{A} - \mathbf{P}\mathbf{B}\mathbf{R}^{-1} \mathbf{B}^\top \mathbf{P} + \mathbf{Q} = 0,
\end{equation}

where $\mathbf{P}$ is the positive semi-definite solution to the ARE.

Once $\mathbf{P}$ is computed, the feedback gain matrix $K$ is:
\begin{equation}
    \mathbf{K} = \mathbf{R}^{-1} \mathbf{B}^\top \mathbf{P}.
\end{equation}

The closed-loop system dynamics under the LQR controller are:
\begin{equation}
    \mathbf{\dot{x}}(t) = \mathbf{(A - B K)}\mathbf{x}(t).
\end{equation}

\paragraph{Design} In order to develop an efficient controller, a great attention has been posed on the estimation of the matrices $\mathbf{Q}$ and $\mathbf{R}$. As far as concerned the $\mathbf{Q}$ matrix, the main relevance was attributed on the values that influence the state position and a moderate relevance on the values that influence the control input. Moreover, some values for $R$ have been estimated considering inherent literature parameters.

\begin{equation}
    \mathbf{Q} =
    \begin{bmatrix}
        25e^3 & 0 & 0 \\ 0 & 0 & 0\\ 0 & 0 & 16e^{-2}
    \end{bmatrix}
    \quad
    \mathbf{R} = 0.5
\end{equation}

Under these assumptions the following gain matrix has been computed.
\begin{equation}
    \mathbf{K} =
    \begin{bmatrix}
        -371.72 & -7.53 & 1.53
    \end{bmatrix}
\end{equation}

The poles of the system have been computed in order to analyze its stability:

\begin{equation}
    eig\mathbf{(A-BK)} =
    \begin{bmatrix}
        -47.51 + 52.95i \\
        -47.51 - 52.95i \\
        -92.90 + 0i
    \end{bmatrix}
\end{equation}

Since eigenvalues of matrix $\mathbf{(A-BK)}$ are situated in left hand-side plan, the resulting system is stable. Neverthless, this control strategy has some limitations that restrict its effectiveness. Indeed, LQR does not inherently provide steady-state error correction for systems with constant disturbances or setpoint changes. It was thus impossible to make the system follows a reference input as in the other examples. In order to do that, some extensions of this simpler controller have been developed and reported in the next sections.

\subsubsection{LQR with tracking capabilities}
\label{subsubsec:lqr_tracking}

The Linear Quadratic Regulator (LQR) with tracking capabilities extends the classical LQR framework to manage systems where the goal is not only to stabilize the system but also to ensure it follows a desired trajectory or reaches a specified setpoint. This advanced control strategy is particularly useful in applications involving reference tracking, where the control objective dynamically changes over time.

To account for tracking, the state-space representation is augmented to include the tracking error:

\begin{equation}
    \mathbf{z} = \begin{bmatrix} \mathbf{x} \\ \mathbf{e} \end{bmatrix},
\end{equation}

where $\mathbf{x}$ is the system state vector, and $\mathbf{e}$ is the error between the system state and the desired reference trajectory.

The cost function for the LQR with tracking is defined as:

\begin{equation}
    \mathcal{J} = \int_0^\infty (\mathbf{z}^\top \mathbf{Q_z} \mathbf{z} + \mathbf{u}^\top \mathbf{R} \mathbf{u} ) dt,
\end{equation}

where $\mathbf{Q_z}$ is the positive semi-definite weighting matrix for the augmented state, and $\mathbf{R}$ is a positive definite weighting matrix for the control input.

The augmented system dynamics are given by:

\begin{equation}
    \mathbf{\dot{z}} = \mathbf{A_z} \mathbf{z} + \mathbf{B_z} \mathbf{u},
\end{equation}

where $\mathbf{A_z}$ and $\mathbf{B_z}$ are derived from the original state-space model:

\begin{equation}
    \mathbf{A_z} = \begin{bmatrix} A & 0 \\ -A_{\text{ref}} & 0 \end{bmatrix}, \quad
    \mathbf{B_z} = \begin{bmatrix} B \\ 0 \end{bmatrix}.
\end{equation}

The optimal control law is derived as:

\begin{equation}
    \mathbf{u}(t) = -\mathbf{K} \mathbf{z}(t),
\end{equation}

where $\mathbf{K}$ is the feedback gain matrix, computed by solving the Riccati equation for the augmented system:

\begin{equation}
    \mathbf{P_z} \mathbf{A_z} + \mathbf{A_z}^\top \mathbf{P_z} - \mathbf{P_z} \mathbf{B_z} \mathbf{R}^{-1} \mathbf{B_z}^\top \mathbf{P_z} + \mathbf{Q_z} = 0.
\end{equation}

This approach ensures accurate reference tracking, balances control effort and tracking performance through the tuning of $\mathbf{Q_z}$ and $\mathbf{R}$, and is robust to disturbances and modeling inaccuracies.

\paragraph{Design} The LQR with reference tracking has been developed using the same matrices $\mathbf{Q}$ and $\mathbf{R}$ described in Subsection \ref{subsubsec:lqr}, with the addiction of the part related to the tracking error.

\paragraph{Step Response} The experimental data measured from system controlled by LQR tracking are reported below. The reference state is followed observing a small gap after the application of the step signal.

\begin{figure}[H]
    \centering
    \includegraphics[width=1\linewidth]{./img/MATLAB/results/step_LQR_tracking_KF.pdf}
    \caption{Step Response}
    \label{fig:Step Response}
\end{figure}

\subsubsection{LQI}
\label{subsubsec:lqi}
The Linear Quadratic Integrator (LQI) is an extension of the classical LQR to achieve reference tracking and disturbance rejection by augmenting the system with integral states. Below are the key equations involved.

The augmented state-space model includes the integral of the output error:

\begin{equation}
    \begin{aligned}
        \mathbf{\dot{x}}(t) = \mathbf{A} \mathbf{x}(t) + \mathbf{B} \mathbf{u}(t), \\ \mathbf{\dot{z}}(t) = \mathbf{C} \mathbf{x}(t) - \mathbf{r}(t),
    \end{aligned}
\end{equation}

where $\mathbf{x}(t)$ is the state vector, $\mathbf{u}(t)$ is the control input, $\mathbf{r}(t)$ is the reference signal, and $\mathbf{z}(t)$ is the integral of the tracking error.

The augmented system can be written as:
\begin{equation}
    \begin{bmatrix}
        \mathbf{\dot{x}}(t) \\
        \mathbf{\dot{z}}(t)
    \end{bmatrix}
    =
    \begin{bmatrix}
        \mathbf{A} & 0 \\
        \mathbf{C} & 0
    \end{bmatrix}
    \begin{bmatrix}
        \mathbf{x}(t) \\
        \mathbf{z}(t)
    \end{bmatrix}
    +
    \begin{bmatrix}
        \mathbf{B} \\
        0
    \end{bmatrix}
    \mathbf{u}(t).
\end{equation}

The cost function to be minimized is:
\begin{equation}
    J = \int_0^\infty \left(
    \begin{bmatrix}
            \mathbf{x}(t) \\
            \mathbf{z}(t)
        \end{bmatrix}^\top
    \begin{bmatrix}
            \mathbf{Q_x} & 0            \\
            0            & \mathbf{Q_z}
        \end{bmatrix}
    \begin{bmatrix}
            \mathbf{x}(t) \\
            \mathbf{z}(t)
        \end{bmatrix}
    + \mathbf{u}(t)^\top \mathbf{R} \mathbf{u}(t)
    \right) dt,
\end{equation}
where $\mathbf{Q_x}$ is the state weighting matrix, $\mathbf{Q_z}$ is the integral state weighting matrix, and $\mathbf{R}$ is the control effort weighting matrix. The optimal control input is:

\begin{equation}
    \mathbf{u}(t) = -\mathbf{K}
    \begin{bmatrix}
        \mathbf{x}(t) \\
        \mathbf{z}(t)
    \end{bmatrix},
\end{equation}

where $\mathbf{K}$ is the feedback gain matrix obtained from solving the Algebraic Riccati Equation (ARE) for the augmented system.

The feedback gain $\mathbf{K}$ is partitioned as:
\begin{equation}
    \mathbf{K} =
    \begin{bmatrix}
        \mathbf{K_x} & \mathbf{K_z}
    \end{bmatrix},
\end{equation}
where $\mathbf{K_x}$ corresponds to the state feedback, and $\mathbf{K_z}$ corresponds to the integral action.

\paragraph{Design} As for the LQR, the $\mathbf{Q}$ and the $\mathbf{R}$ matrices have been estimated in order to implement the LQI controller. The weighting parameters are the same except for the additional contribution which characterizes a huge relevance on the tracking error.

\begin{equation}
    \begin{aligned}
        \mathbf{Q} & =
        \begin{bmatrix}
            25e^3 & 0 & 0        & 0      \\
            0     & 0 & 0        & 0      \\
            0     & 0 & 16e^{-2} & 0      \\
            0     & 0 & 0        & 10^{6}
        \end{bmatrix},  \quad
        \mathbf{R} & = 0.5.
    \end{aligned}
\end{equation}

The feedback gain matrix  $\mathbf{K}$ is thus been computed:

\begin{equation}
    \begin{aligned}
        \mathbf{K} & =
        \begin{bmatrix}
            -513.31 & -9.19 & 1.71 & 4472.13
        \end{bmatrix}.
    \end{aligned}
\end{equation}

Finally, the following eigenvalues have been computed to ensure control stability:

\begin{equation}
    \begin{aligned}
        eig\mathbf{(A-BK)}  =
        \begin{bmatrix}
            -19.74 + 0i     \\
            -46.54 + 53.49i \\
            -46.54 - 53.49i \\
            -92.40 + 0i
        \end{bmatrix}.
    \end{aligned}
\end{equation}

\paragraph{Step Response} The measured data are reported in Figure \ref{fig:Step Response} to exhibit the behavior of the system state correspondent to the application of a step signal.

\begin{figure}[H]
    \centering
    \includegraphics[width=1\linewidth]{./img/MATLAB/results/step_LQI_KF.pdf}
    \caption{Step Response}
    \label{fig:Step Response}
\end{figure}

\subsection{MPC Controllers}
\label{subsec:mpc_controllers}

\subsubsection{MPC with linear model}
\label{subsubsec:mpc_linear}