\subsection{Extended Kalman Filter}
\label{subsec:extended_kalman_filter}

The extended Kalman filter (EKF) is the nonlinear version of the Kalman filter which linearizes about an estimate of the current mean and covariance.
Same assumptions done for the Kalman filter hold for the EKF, expect for the linearity of the system model.

The extended Kalman filter is a recursive state estimator for nonlinear systems.
It recomputes on the fly (online) the system matrices based on the current state estimate.
Once the system matrices are recomputed, the Kalman gain is computed again based on Equation \ref{eq:kalman_gain}, where now $P$ is the solution to the Riccati equation:

\begin{equation}
    \dot{P} = A P + P A^T - P C^T (C P C^T + R)^{-1} C P + Q
    \label{eq:riccati_equation}
\end{equation}

Notice that, unlike its linear counterpart, the extended Kalman filter it's not inherently an optimal estimator.
It only becomes optimal in cases where both the measurement model and the state transition model are linear, as under those conditions, the EKF effectively reduces to the standard Kalman filter.

Furthermore, if the initial state estimate is inaccurate or if the process is poorly modeled, the EKF may diverge rapidly because of the inherent approximations introduced by its linearization approach.

Another challenge associated with the EKF is that its estimated covariance matrix often underrepresents the true covariance matrix.
This underestimation can lead to statistical inconsistency unless corrective measures are applied, such as the introduction of "stabilizing noise."

\paragraph{Design}

The design of the EKF is similar to the Kalman filter with the key difference that the gain matrix $K_k$ is computed online at each time step $t_k$ based on the current state estimate $\hat{x}_k$.
