\section{Controllers design}
\label{sec:controllers_design}

\begin{frame}{Overview}
    The control of the system is achieved by designing different controllers:
    \begin{itemize}
        \item \textbf{PIDs controller};
        \item \textbf{LQs controller};
        \item \textbf{MPC controller}.
    \end{itemize}
\end{frame}



\begin{frame}{PID with Anti-Windup}
    \normalsize
    \begin{itemize}
        \item \textbf{Problem}: \textbf{Integrator windup} occurs when the actuator saturates, causing uncontrolled growth of the integral term. This degrades the rise time and can lead to a \textbf{higher overshoot}.
        \item \textbf{Solution}: \textbf{Conditional Integration}. The controller output is compared with the actuator limits. If saturation is detected and error accumulates, the integrator is \textbf{turned off}.
    \end{itemize}

    The gain parameters reported below have been estimated firstly considering the Bode diagram, and subsequently analyzing the Root Locus:
    \begin{equation}
        K_p = -150 \quad K_i = -450 \quad K_d = -6.82
    \end{equation}
\end{frame}



\begin{frame}{PID with Gain Scheduling}

    \begin{itemize}
        \item \textbf{PID controller} tuned at a series of \textbf{steady-state operating points}.
        \item The sphere's movement space is divided into \textbf{multiple points}, and the system is \textbf{linearized} locally at each operating point.
        \item Controller gains are \textbf{optimized} for each operating condition.
        \item Gain curves \textbf{gradually vary} between points.
    \end{itemize}

    \begin{table}[H]
        \centering
        \begin{tabular}{|c|c|c|c|}
            \hline
            $z [mm]$ & $K_p$  & $K_i$   & $K_d$   \\
            \hline
            $5$      & $-102$ & $-306$  & $-4.64$ \\
            $8$      & $-136$ & $-408$  & $-6.18$ \\
            $12$     & $-183$ & $-550$  & $-8.34$ \\
            $16$     & $-250$ & $-750$  & $-11.4$ \\
            $20$     & $-342$ & $-1030$ & $-15.5$ \\
            \hline
        \end{tabular}
        \caption{Look up table for the PID gain scheduling}
        \label{tab:pid_gain_scheduling_gains}
    \end{table}
\end{frame}



\begin{frame}{LQR with tracking capabilities}

    \textbf{The LQR with tracking capabilities}, extends the classical LQR framework to manage systems where the goal is not only to stabilize the system but also to ensure it follows a desired trajectory.

    For the matrix $\mathbf{Q}$, particular attention is given to the values affecting the position of the sphere, with moderate emphasis on the current state.
    The matrix $\mathbf{R}$ is estimated based on literature parameters.

    \begin{itemize}
        \item \textbf{Design parameters:}
              \begin{equation}
                  \mathbf{Q} =
                  \begin{bmatrix}
                      25\cdot10^3 & 0 & 0              \\
                      0           & 0 & 0              \\
                      0           & 0 & 16\cdot10^{-2}
                  \end{bmatrix}
                  \quad
                  \mathbf{R} = 0.5
              \end{equation}

        \item \textbf{Computed feedback gain:}
              \begin{equation}
                  \mathbf{K} =
                  \begin{bmatrix}
                      -371.72 & -7.53 & 1.53
                  \end{bmatrix}
              \end{equation}
    \end{itemize}
\end{frame}



\begin{frame}{LQI}

    \textbf{The LQI (Linear Quadratic Integrator)} is an extension of LQR that incorporates integral states to improve reference tracking and disturbance rejection.
    By adding integration of tracking error, LQI reduces persistent errors between the system and reference, improving control accuracy.

    \begin{itemize}
        \item \textbf{Design parameters:}
              \begin{equation}
                  \mathbf{Q} = \begin{bmatrix}
                      25\cdot10^3 & 0 & 0              & 0    \\
                      0           & 0 & 0              & 0    \\
                      0           & 0 & 16\cdot10^{-2} & 0    \\
                      0           & 0 & 0              & 10^6
                  \end{bmatrix} \quad \mathbf{R} = 0.5
              \end{equation}
        \item \textbf{Computed feedback gain:}
              \begin{equation}
                  \mathbf{K} = \begin{bmatrix}
                      -513.31 & -9.19 & 1.71 & 4472.13
                  \end{bmatrix}
              \end{equation}
    \end{itemize}
\end{frame}



\begin{frame}{MPC}

    \textbf{Model Predictive Control (MPC)} is an advanced control strategy that optimizes system performance by continuously reinitializing the control procedure.

    \begin{itemize}
        \item Based on a linear system model.
        \item Constraints are applied.
        \item Computationally less expensive compared to nonlinear MPC.
    \end{itemize}

    \begin{equation}
        \begin{aligned}
            \textbf{Prediction Horizon} & = 0.1 \, \text{s}  \\
            \textbf{Control Horizon}    & = 0.01 \, \text{s}
        \end{aligned}
    \end{equation}

    Constraints on position and control are applied, as shown in the table below:

    \begin{table}[H]
        \centering
        \begin{tabular}{|c|c|c|}
            \hline
            \textbf{Variable} & \textbf{Max} & \textbf{Min} \\
            \hline
            Position          & 20 mm        & 0 mm         \\
            \hline
            Control           & 1            & 0            \\
            \hline
        \end{tabular}
        \caption{Constraints for the MPC controller}
    \end{table}
\end{frame}
