\section{Conclusions}

\begin{frame}{Conclusions}

    % The work carried out in this project was always based on the linearized model of the MagLev system.
    Ball levitation was achieved by implementing various control strategies and filtering methods, and the performance of each was evaluated through simulations and real-world experiments.

    \vspace{9pt}

    Based on the results obtained, we can state that:

    \begin{itemize}
        \item For \textbf{control strategies}, the \textbf{LQI Controller} emerged as the most effective, providing accurate control with minimal oscillations and high stability.
        \item Regarding \textbf{filtering \& estimator methods}, the \textbf{Kalman Filter} demonstrated the best performance, ensuring accurate state estimation and disturbance rejection.
    \end{itemize}

\end{frame}



\begin{frame}{Future work}

    For future work, we suggest the implementation of control techniques that consider the full nonlinear model of the system, such as:

    \begin{itemize}
        \item \textbf{Nonlinear Model Predictive Control}
        \item \textbf{Feedback Linearization}
        \item \textbf{Backstepping controllers}
    \end{itemize}

    These methods could improve the system's robustness and adaptability, offering better performance in handling larger disturbances, nonlinearities, and uncertainties.

\end{frame}
