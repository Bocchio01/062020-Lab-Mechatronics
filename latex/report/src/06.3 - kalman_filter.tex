\subsection{Kalman Filter}
\label{sec:kalman_filter}

% The Kalman filter is a powerful tool used to estimate, predict and filter the states of a linear dynamic system.

% The Kalman filter is based on the following assumptions:

% \begin{itemize}
%     \item The system is linear.
%     \item The system is subject to Gaussian noise.
%     \item The noise is white and has a known covariance.
% \end{itemize}

% When the above assumptions are valid, we can use the Kalman filter algorithm to compute the optimal observer gain matrix $K$ that allows to estimate the states of the system.

The Kalman Filter is a powerful algorithm used for estimating the state of a dynamic system from a series of noisy measurements.
It is widely used in control systems, robotics, signal processing, and navigation due to its ability to provide real-time, optimal state estimates by considering system dynamics and measurement noise.

The filter operates in two main steps: the prediction step, which estimates the current state based on the system model, and the update step, which refines this estimate using new measurement data.


Mathematically, the filter assumes a linear system model of the form:

\begin{equation}
    \begin{aligned}
        \dot{x} & = A x + B u + w \\
        y       & = C x + D u + v
    \end{aligned}
\end{equation}

Where $x$ is the state vector, $u$ is the control input, $y$ is the measurement, $A$ is the state transition matrix, $B$ is the control input matrix, $C$ is the observation matrix, and $w$ and $v$ are process and measurement noise, respectively.
Among the assumptions of the Kalman filter, both the process and measurement noise are assumed to be zero-mean Gaussian white noise with known covariances $Q$ and $R$.

One of the key strengths of the Kalman Filter is its ability to provide smooth state estimates, even in the presence of high measurement noise, without introducing delays.
This makes it critical for real-time applications such as tracking, navigation (e.g., GPS), and autonomous systems.
Additionally, it assumes the noise covariances $Q = \text{Cov}(w_k)$ and $R = \text{Cov}(v_k)$ are known, which helps in calculating the Kalman gain:

\begin{equation}
    K = P^- H^T (H P^- H^T + R)^{-1}
    \label{eq:kalman_gain}
\end{equation}

Where $P_k^-$ is the predicted estimate covariance.

Once the Kalman gain is calculated, the filter proceeds to the update step, where it corrects the predicted state estimate based on the measurement.
To do so, the same structure seen already in the Luenberger observer is used (Equation \ref{eq:observer_dynamics}), but with the Kalman gain instead of the observer gain.
This reads:

\begin{equation}
    \begin{cases}
        \dot{\hat{x}} = A \hat{x} + B u + K(y - C \hat{x}) \\
        \hat{y} = C \hat{x}
    \end{cases}
\end{equation}


\paragraph{Design}

As we have already seen in the Luenberger observer design, the poles of the observer are given by the eigenvalues of the matrix $A - KC$, and the observer is stable if the poles are placed in the left half plane of the complex plane.
The Kalman gain $K$ can be computed using Equation \ref{eq:kalman_gain}.

By doing so, we obtain the following $K$ matrix:

\begin{equation}
    K = \begin{bmatrix}
        ?? \\
        ?? \\
        ??
    \end{bmatrix}
    \label{eq:kalman_gain_matrix}
\end{equation}

