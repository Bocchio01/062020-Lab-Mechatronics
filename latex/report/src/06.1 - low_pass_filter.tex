\subsection{Low Pass Filter}
\label{sec:low_pass_filter}

\hl{To be checked. Not sure about the correctness of the calculations and the estimation of the natural frequency of the system.}

The low pass filter is a filter that allows the low frequency components of a signal to pass through, while attenuating the high frequency components.
By correctly choosing the cut-off frequency of the filter, it's possible to remove the noise from the signal, while preserving the useful information.

The transfer function of a first order low pass filter is given by:

\begin{equation}
    G(s) = \frac{1}{\tau s + 1}
\end{equation}

Where $\tau$ is the time constant of the filter, and it's related to the cut-off frequency $\omega_c$ by the relation $\tau = \frac{1}{\omega_c}$.

\paragraph{Filter on position}

From the \texttt{Inteco} manual, we have understood that the vertical velocity of the ball is computed via numerical discretization of the position.
This also means that the noise present in the position measurement is amplified by the differentiation process.
To reduce this noise, we design a low pass filter to be applied to the position measurements before the differentiation.

In order to estimate the cut-off frequency of the filter, one can either perform a spectral decomposition of the signal to obtain the maximum frequency present in the signal, or estimate (via experiments on the real system) the natural frequency of the system and choose a cut-off frequency accordingly.

From preliminary tests, we have observed the ball reaches maximum speeds of around $v_{max} \approx 0.3 m/s$.
Therefore, we can roughly the period of the ball to be around $T_p = 2 \frac{(h-2r)}{v_{max}} \approx 0.245s$.
From this, we can estimate the natural frequency of the system to be:

\begin{equation}
    \omega_n = \frac{2\pi}{T_p} \approx 25 rad/s
\end{equation}

As rule of thumb, we could choose the cut-off frequency to be one decade after the natural frequency, i.e. $\omega_c = 10 \omega_n \approx 250 rad/s$.

However, we know that the smaller the cut-off frequency, the better the noise attenuation.
On the other hand, the smaller the cut-off frequency, the larger the phase lag introduced by the filter.

As a compromise between noise attenuation and phase lag, we choose the cut-off frequency to be $\omega_c = 200 rad/s$.

By doing so, we obtain the time constant of the filter to be $\tau = \frac{1}{200} = 5 ms$ and a corresponding phase delay introduced by the filter of:

\begin{equation}
    \phi = -\arctan{(\omega_n \tau)} = -\arctan{(25 \cdot 5 \cdot 10^{-3})} \approx -7.1^{\circ}
\end{equation}


\paragraph{Filter on current}

The current measurement is also affected by noise.
However, based on experiments, we have observed that even a slight delay in the current measurement can lead to instability of the system.
Therefore, we choose not to apply a low pass filter to the current measurements.

