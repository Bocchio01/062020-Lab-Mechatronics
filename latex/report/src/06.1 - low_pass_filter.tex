\subsection{Low Pass Filter}
\label{sec:low_pass_filter}

The low pass filter is a filter that allows the low frequency components of a signal to pass through, while attenuating the high frequency components.
By correctly choosing the cut-off frequency of the filter, it's possible to remove the noise from the signal, while preserving the useful information.

The transfer function of a first order low pass filter is given by:

\begin{equation}
    G(s) = \frac{1}{\tau s + 1}
\end{equation}

Where $\tau$ is the time constant of the filter, and it's related to the cut-off frequency $\omega_c$ by the relation $\tau = \frac{1}{\omega_c}$.

\paragraph{Filter on position}

From the \texttt{Inteco} manual, we have understood that the vertical velocity of the ball is computed via numerical discretization of the position.
This also means that the noise present in the position measurement is amplified by the differentiation process.
To reduce this noise, we design a low pass filter to be applied to the position measurements before the differentiation.

From the Bode plot of Figure \ref{fig:bode_plot}, we can observe that the bandwidth of system (computed as the frequency at which a reduction of $3dB$ happens) is around:

\begin{equation}
    \omega_n = \frac{2\pi}{T_p} \approx 20 rad/s
\end{equation}

As rule of thumb, we choose the cut-off frequency of the filter to be one decade after the natural frequency of the system, i.e. $\omega_c = 10 \omega_n \approx 200 rad/s$.

By doing so, we obtain the time constant of the filter to be $\tau = \frac{1}{200} = 5 ms$ and a corresponding phase delay of:

\begin{equation}
    \phi = -\arctan{(\omega_n \tau)} = -\arctan{(20 \cdot 5 \cdot 10^{-3})} \approx -5.7^{\circ}
\end{equation}


\paragraph{Filter on current}

The current measurement is also affected by noise.
However, based on experiments, we have observed that even a slight delay in the current measurement can lead to instability of the system.
Therefore, we choose not to apply a low pass filter to the current measurements.

