\subsection{Controllability and observability}
\label{subsec:controllability_observability}

The controllability and observability of the system are crucial aspects to consider when designing a control strategy.
Controllability ensures that the system's state can be manipulated by the control inputs, while observability guarantees that the state can be accurately estimated from the system's outputs.

The controllability matrix $\mathcal{KR}$ and observability matrix $\mathcal{KO}$ are defined as follows:

\begin{equation}
    \begin{aligned}
        \mathcal{KR} & =
        \begin{bmatrix}
            B & AB & A^2B
        \end{bmatrix}   \\
        \mathcal{KO} & =
        \begin{bmatrix}
            C^T & (CA)^T & (CA^2)^T
        \end{bmatrix}
    \end{aligned}
\end{equation}

By computing the rank of the controllability and observability matrices, we can determine whether the system is controllable and observable.
In particular, based on the Kalman's reachability and observability conditions, the system is controllable if and only if $\text{rank}(\mathcal{KR}) = n$ and observable if and only if $\text{rank}(\mathcal{KO}) = n$, where $n$ is the number of states in the system.

An explicit computation shows that the system is both controllable and observable, given that:

\begin{equation}
    \mathcal{KR} = 10^{5}
    \begin{bmatrix}
        0      & 0       & -0.0178 \\
        0      & -0.0178 & 0.6250  \\
        0.0010 & -0.0346 & 1.2180
    \end{bmatrix}
    \quad
    \mathcal{KO} = 10^{3}
    \begin{bmatrix}
        0.0010 & 0      & 0       \\
        0      & 0.0010 & 0       \\
        1.8025 & 0      & -0.0181
    \end{bmatrix}
\end{equation}
