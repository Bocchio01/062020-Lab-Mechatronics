\subsection{PD Controller}
\label{sec:pd}

The Proportional-Derivative (PD) controller is a simple controller that uses the error signal and its derivative to compute the control signal.
Its derived from the Proportional-Integral-Derivative (PID) controller, which is one of the most common controllers used in industry, with a control signal given by:

\begin{equation}
    u(t) = K_p e(t) + K_i \int_{0}^{t} e(\tau)dt + K_d \frac{de(t)}{dt} = K_p \left(e(t) + \frac{1}{T_i} \int_{0}^{t} e(\tau)dt + T_d \frac{de(t)}{dt}\right)
\end{equation}

Where $K_p$, $K_i$ and $K_d$ are the proportional, integral and derivative gains, respectively, and $T_i$ and $T_d$ are the integral and derivative time constants, respectively.

Given that our system doesn't have a friction term at steady state condition, the integral action that ensures zero steady-state error is not necessary and can be omitted (i.e. $K_i = 0$).

In order to controll the \acrshort{mls} system, we need to linearize the system around a desired operating point, and tune the controller gains accordingly.

\paragraph{Linearization \& State space representation}

The linearization of the system is done by computing the Jacobian matrix of the system around the desired operating point.

At the end of the process, we obtain the state space representation of the system:

\begin{equation}
    \begin{cases}
        \dot{\mathbf{x}} = A \mathbf{x} + B \mathbf{u} \\
        y = C \mathbf{x}
    \end{cases}
\end{equation}

We can define the state vector $\mathbf{x}$ and the input vector $\mathbf{u}$ as:

\begin{equation}
    \mathbf{x} = \begin{bmatrix}
        z   \\
        v   \\
        I_1 \\
        I_2
    \end{bmatrix}
    \quad
    \mathbf{u} = \begin{bmatrix}
        0   \\
        0   \\
        V_1 \\
        V_2
    \end{bmatrix}
\end{equation}

The matrices $A$, $B$ and $C$ are then defined as:

\begin{equation}
    A = \begin{bmatrix}
        ? & ? & ? & ? \\
        ? & ? & ? & ? \\
        ? & ? & ? & ? \\
        ? & ? & ? & ?
    \end{bmatrix}
    \quad
    B = \begin{bmatrix}
        0 \\
        0 \\
        ? \\
        ?
    \end{bmatrix}
    \quad
    C = \begin{bmatrix}
        1 & 0 & 0 & 0
    \end{bmatrix}
\end{equation}
