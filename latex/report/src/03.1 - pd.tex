\subsection{PD Controller}
\label{sec:pd}

The Proportional-Derivative (PD) controller is a simple controller that uses the error signal and its derivative to compute the control signal.
Its derived from the Proportional-Integral-Derivative (PID) controller, which is one of the most common controllers used in industry, with a control signal given by:

\begin{equation}
    u(t) = K_p e(t) + K_i \int_{0}^{t} e(\tau)dt + K_d \frac{de(t)}{dt} = K_p \left(e(t) + \frac{1}{T_i} \int_{0}^{t} e(\tau)dt + T_d \frac{de(t)}{dt}\right)
\end{equation}

Where $K_p$, $K_i$ and $K_d$ are the proportional, integral and derivative gains, respectively, and $T_i$ and $T_d$ are the integral and derivative time constants, respectively.

Given that our system doesn't have a friction term at steady state condition, the integral action that ensures zero steady-state error is not necessary and can be omitted (i.e. $K_i = 0$).

In order to controll the \acrshort{mls} system, we need to linearize the system around a desired operating point, and tune the controller gains accordingly.

