\subsection{Force analysis}
\label{subsec:force_analysis}

Thanks to the data obtained from the previous tests, we are already able to predict the force applied to the ball by the inductance.
In particular, we already know that the electromagnetic force applied to the ball is given by the following equation:

\begin{equation}
    F_{em} = \frac{1}{2} \frac{\partial L}{\partial z} I^2 = \frac{1}{2} (-a_z L_z e^{-a_z z}) I^2
\end{equation}

Because of the previously identified parameters, we have an analytical expression for the sensitivity of the inductance with respect to the position of the ball.
However, due to uncertainties in the identification of the parameters, we can expect some discrepancies between the predicted force and the measured one.

In order to quantify these discrepancies and validate the model, we use a direct method to measure the force applied to the ball by the inductance and compare it with the predicted one.
To do so, we recall Equation \ref{eq:reduced_equations_of_motion_final} and in particular the equation relative to $\dot{v}$:

\begin{equation}
    \dot{v} = m^{-1} \left(\frac{1}{2} \frac{\partial L_1}{\partial z} I_1^2 + \frac{1}{2} \frac{\partial L_2}{\partial z} I_2^2 + m g  \right)
\end{equation}

If we consider the system at rest or equivalently at the incipient motion of the ball, we can simplify the equation as follows:

\begin{equation}
    0 = \frac{1}{2} \frac{\partial L_1}{\partial z} I_1^2 + \frac{1}{2} \frac{\partial L_2}{\partial z} I_2^2 + m g
\end{equation}

Supposing now that only the first coil is energized, we can further simplify the equation as follows:

\begin{equation}
    0 = \frac{1}{2} \frac{\partial L_1}{\partial z} I_1^2 + m g
\end{equation}

Which leads to:

\begin{equation}
    \frac{\partial L_1}{\partial z} = -2 \frac{m g}{I_1^2}
    \label{eq:sensitivity_of_inductance}
\end{equation}

This last equation basically tells us that in steady state conditions, when the ball is levitating (i.e. $\dot{z} = 0$ and not supported by any platform), the sensitivity of the inductance of the first coil has an analytical expression that can be directly evaluated by measuring the current in the first coil and the position of the ball.

In order to follow this approach, the experimental steps are as follows:

\begin{enumerate}
    \item By regulating a lower platform, the ball is placed at a certain height ($z^*$);
    \item A linearly increasing voltage is applied to the first coil;
    \item The current circulating in the first coil is measured;
    \item The current at which the ball starts to levitate is identified;
    \item The sensitivity of the inductance is calculated using Equation \ref{eq:sensitivity_of_inductance}.
    \item The test is repeated for different initial positions of the ball.
\end{enumerate}

In Figure \ref{fig:levitation_current} we can see both the position of the ball (red line) and the current circulating in the first coil (black line) around the identified levitation point (marked by the vertical black dashed line).

\begin{figure}[H]
    \centering
    \includegraphics[width=1\textwidth]{img/MATLAB/identification/currents_for_force.pdf}
    \caption{Position of the ball and current in the first coil around the levitation point (marked by vertical black dashed line)}
    \label{fig:levitation_current}
\end{figure}

Instead, in Figure \ref{fig:dynamic_inductance_characteristics}, we can observe both the measured data and the fitted ones.
On the right side figure, a complete characterization of the electromagnetic force has been reconstructed based again on the above equations.

\begin{figure}[H]
    \centering
    \includegraphics[width=1\textwidth]{img/MATLAB/identification/force.pdf}
    \caption{Dynamic inductance characteristics and electromagnet force}
    \label{fig:dynamic_inductance_characteristics}
\end{figure}

The left-hand side of Figure \ref{fig:dynamic_inductance_characteristics} shows a comparison between the measured data (black circles), their fitting (red line) and the sensitivity of the inductance coming from the parameters identified in Section \ref{subsec:inductances_characterization} (yellow line).
Data shows great accuracy in almost the entire range of the ball position.

The right-hand side of Figure \ref{fig:dynamic_inductance_characteristics} shows the electromagnetic force generated by the first coil as a function of both the ball position and the current circulating in the coil.
One can notice that the force has an exponential behavior with respect to the ball position and a quadratic behavior with respect to the current.