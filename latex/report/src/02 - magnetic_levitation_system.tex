\section{Magnetic Levitation System}
\label{sec:magnetic_levitation_system}

As stated in the introduction, the system under study is the \acrfull{mls} provided by \texttt{Inteco} (producer website: \url{https://www.inteco.com.pl/products/magnetic-levitation-systems/}).
In Figure \ref{fig:MLS} the system used in this work is shown.

\begin{figure}[H]
    \centering
    \includegraphics[width=0.7\textwidth]{./img/maglev_and_components.jpg}
    \caption{\acrlong{mls}}
    \label{fig:MLS}
\end{figure}

As it can be seen quite clearly, the system is composed of a simple mechanical structure that is used to support two electromagnets and an optical infrared sensor.
Along with the mechanical structure, a ferromagnetic ball and a control unit are present.

At its core principle, the system uses the interaction between the magnetic field generated by the electromagnets and the ferromagnetic ball to keep the ball in a desired position.
The optical sensor is used to measure the position of the ball and provide feedback to the control unit that, in turn, adjusts the voltage applied to (and indeed the current flowing through) the electromagnets to keep the ball in a desired position.
In Figure \ref{fig:MLS_general_scheme} a schematic representation of the upper half of the system is shown.

\begin{figure}[H]
    \centering
    \includegraphics[width=0.9\textwidth]{./img/maglev_general_scheme.png}
    \caption{Schematic representation of the upper half of the \acrshort{mls} system.}
    \label{fig:MLS_general_scheme}
\end{figure}

\paragraph{Real world application}

Despite the fact that our system is a simplified use case, the magnetic levitation principle is used in many real-world applications.

One of the most common applications is the magnetic levitation trains, also known as `MagLev' trains.
These trains use the magnetic levitation principle to lift the train off the tracks and propel it forward using the magnetic field generated by the tracks.
The main advantage of this technology is the absence of friction between the train and the tracks, which allows the train to reach higher speeds and reduce the noise characteristic of the traditional trains.
Some of the fastest (operating) trains in the world are MagLev trains, with the Shanghai MagLev train being the fastest, reaching a top speed of $623 km/h$ \cite{WikiSCMaglev}.

Another application of the magnetic levitation principle is the magnetic bearings.
These bearings use the magnetic field generated by electromagnets to levitate a rotor and keep it in a desired position.
The main advantage of this technology is the absence of mechanical contact between the rotor and the stator, which allows the rotor to reach higher speeds and reduce the wear of the components.