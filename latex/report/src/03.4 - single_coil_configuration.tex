\subsection{Single Coil Configuration}
\label{subsec:single_coil_configuration}

In this section, we will present the model of the \acrshort{mls} system when only the upper coil is used for control purposes.
This configuration is the one that will be used in the rest of the document and the laboratory activities.
The choice has been taken in order to deal with a simpler SISO system, which is easier to control and analyze.

In the following, starting from Equation \ref{eq:reduced_equations_of_motion_final}, we will at derive the reduced model, linearize it and represent it in state-space form.

\paragraph{Reduced Equations of Motion}

At first, if we consider energizing only the upper coil, we can simply remove the terms related to the lower coil from the equations of motion.
Based on Equation \ref{eq:reduced_equations_of_motion_final}, we can write the following equations:

\begin{equation}
    \begin{cases}
        \dot{z} = v                                                                             \\
        \dot{v} = m^{-1} \left(\frac{1}{2} \frac{\partial L_1}{\partial z} I_1^2 + m g  \right) \\
        \dot{I_1} = L_1^{-1} \left(- R_1 I_1 + (k_1 U_1 + c_1) \right)
    \end{cases}
    \label{eq:equations_of_motion_single_coil}
\end{equation}

\paragraph{Linearization}

As already discussed in Section \ref{subsec:model_linearization}, we can linearize via Taylor expansion the equations of motion around one of its operating points.
For the case of the single coil configuration, Equations \ref{eq:operating_point_states} and \ref{eq:operating_point_inputs} reduce to:

\begin{equation}
    \mathbf{x}_{op} =
    \begin{bmatrix}
        z_{op} \\
        v_{op} \\
        I_{1op}
    \end{bmatrix}
    =
    \begin{cases}
        z^* \\
        0   \\
        \sqrt{ -(2m g + \frac{\partial L_2}{\partial z} \big|_{z_{op}} (\frac{V_{2min}}{R_{20}})^2) / \frac{\partial L_1}{\partial z} \big|_{z_{op}} }
    \end{cases}
    \label{eq:operating_point_states_single_coil}
\end{equation}

\begin{equation}
    \mathbf{u}_{op} =
    \begin{bmatrix}
        U_{1op}
    \end{bmatrix}
    =
    \begin{cases}
        \max{\left[0, (R_{10} I_{1op} - c_1) / k_1 \right]}
    \end{cases}
    \label{eq:operating_point_inputs_single_coil}
\end{equation}

By performing the Taylor expansion of Equations \ref{eq:equations_of_motion_single_coil} around the operating point, we obtain the following linearized model:

\begin{equation}
    \begin{cases}
        \dot{z} - \dot{z_{op}}    & \approx 1 (v - v_{op}) \\
        \dot{v} - \dot{v_{op}}    & \approx m^{-1} \left(
        \frac{1}{2} \frac{\partial^2 L_1}{\partial z^2} \Bigg|_{\mathbf{x}_{op}} (z - z_{op}) I_{1op}^2 +
        \frac{\partial L_1}{\partial z} \Bigg|_{\mathbf{x}_{op}} I_{1op} (I_1 - I_{1op})
        \right)                                            \\
        \dot{I_1} - \dot{I_{1op}} & \approx
        \left(- L_1^{-2} \frac{\partial L_1}{\partial z} \left(- R_1 I_1 + k_1 U_1 + c_1 \right) \right) \Bigg|_{\mathbf{x}_{op}} (z - z_{op}) +
        \left(- L_1^{-1} R_1 \right) \Bigg|_{\mathbf{x}_{op}} (I_1 - I_{1op}) +
        \left(L_1^{-1} k_1 \right) \Bigg|_{\mathbf{x}_{op}} (U_1 - U_{1op})
    \end{cases}
    \label{eq:linearized_model_single_coil}
\end{equation}

\paragraph{State-Space Representation}

Finally, we can represent the linearized model in state-space form.
Given the reduction of the system to a SISO one, we need to redefine the state vector $\mathbf{x}$ and the input vector $\mathbf{u}$ as follows:

\begin{equation}
    \mathbf{x} = \begin{bmatrix}
        z \\
        v \\
        I_1
    \end{bmatrix}
    \quad
    \mathbf{u} = \begin{bmatrix}
        U_1
    \end{bmatrix}
\end{equation}


Once the state and input vectors have been defined, the linearized state-space representation can be obtained by leveraging the linearized model derived previously.
The matrices $A$, $B$, $C$ and $D$ are then defined as:

\begin{equation}
    \begin{aligned}
        A & = \frac{\partial f}{\partial \mathbf{x}} \Bigg|_{(\mathbf{x}_{op}, \mathbf{u}_{op})}
        = \begin{bmatrix}
              0      & 1 & 0      \\
              a_{21} & 0 & a_{23} \\
              a_{31} & 0 & a_{33} \\
          \end{bmatrix}                                                                    \\
        B & = \frac{\partial f}{\partial \mathbf{u}} \Bigg|_{(\mathbf{x}_{op}, \mathbf{u}_{op})}
        = \begin{bmatrix}
              0 \\
              0 \\
              b_{31}
          \end{bmatrix}                                                                         \\
        C & = \frac{\partial g}{\partial \mathbf{x}} \Bigg|_{(\mathbf{x}_{op}, \mathbf{u}_{op})}
        = \begin{bmatrix}
              1 & 0 & 0
          \end{bmatrix}                                                                         \\
        D & = \frac{\partial g}{\partial \mathbf{u}} \Bigg|_{(\mathbf{x}_{op}, \mathbf{u}_{op})}
        = \begin{bmatrix}
              0
          \end{bmatrix}
    \end{aligned}
\end{equation}

Given the non-correlation between the two coils currents, the elements of the matrices $A$, $B$, $C$ and $D$ remain exactly as already computed in Section \ref{eq:state_space_matrices}.