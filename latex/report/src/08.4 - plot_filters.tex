\subsection{Comparison of filters}
\label{subsec:plot_filters}

The filters implemented are compared based on the system's response to the fast sinusoidal reference. The controller in the comparison of filters is an LQR tracking controller. The goal is to evaluate the effectiveness and impact of each filtering method on the accuracy of the sphere's trajectory respect to the sinusoidal reference and the quality of the obtained control signal. The compared graphs include the control without filtering, a low-pass filter, a Luenberger observer, a standard Kalman filter, and an extended Kalman filter (EKF).

\textbf{1. Control Without Filters:}
This case represents the least effective way to control the system. Clearly due to the absence of filtering, the system noise is not reduced. Noise filtering, especially for velocity, is critical to achieving accurate response. Using the controller without filters results in imprecise trajectory tracking. The control performance is visibly affected by oscillations and disturbances, making this approach the least suitable for the system under consideration.

\textbf{2. Low Pass Filter:}
Introducing a low pass filter improves the response accuracy compared to the unfiltered case. However, its ability to eliminate noise is limited by the filter's non-adaptive nature, which hinders dynamic performance. The accuracy is higher respect to the case without filters, but lower respect to the other filters, particularly for trajectories requiring rapid variations. To illustrate this behavior, a sinusoidal reference with the minimum period was selected for testing.

\textbf{3. Luenberger Observer:}
The Luenberger observer proves to be more effective than the low pass filter due to its ability to estimate unmeasured system states. However, the control quality remains inferior compared to the Kalman filters. The observer’s performance aligns with expectations, as it does not employ an optimal gain matrix (K) like the Kalman filter.

\textbf{4. Standard Kalman Filter:}
The Kalman filter emerged as the most effective method among those tested. Its ability to optimize state estimation in the presence of measurement and process noise significantly enhances control quality. The trajectory tracking exhibited the highest accuracy with minimal oscillations.

\textbf{5. Extended Kalman Filter:}
Although the EKF is designed to handle system nonlinearities, the results showed inferior performance compared to the standard Kalman filter. Specifically, the EKF failed to adequately follow the reference trajectories. This could be attributed to suboptimal linearizations or poorly tuned covariance matrices. These findings suggest that applying the EKF requires further investigation and optimization to achieve competitive performance.

\textbf{Conclusions:}
The comparative analysis highlighted a clear hierarchy in the performance of the various approaches. The standard Kalman filter stands out as the optimal solution for the analyzed system, ensuring precise and stable control. In contrast, the Luenberger observer and the low-pass filter provide incremental improvements over the unfiltered case but are outperformed by the Kalman filter. Despite its theoretical potential, the EKF delivered unsatisfactory results compared to its simpler counterpart, indicating that it represents a promising area for future research.

These results underscore the importance of selecting appropriate filtering and observation methods in the design of advanced control systems and pave the way for further studies to improve the integration of nonlinear filters in real-world applications.

