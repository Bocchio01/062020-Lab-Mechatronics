\subsection{State Space Representation}
\label{subsec:state_space_representation}

The linearization of the system is done by computing the Jacobian matrix of the system around the desired operating point.

At the end of the process, we obtain the state space representation of the system:

\begin{equation}
    \begin{cases}
        \dot{\mathbf{x}} = A \mathbf{x} + B \mathbf{u} \\
        y = C \mathbf{x}
    \end{cases}
\end{equation}

We can define the state vector $\mathbf{x}$ and the input vector $\mathbf{u}$ as:

\begin{equation}
    \mathbf{x} = \begin{bmatrix}
        z   \\
        v   \\
        I_1 \\
        I_2
    \end{bmatrix}
    \quad
    \mathbf{u} = \begin{bmatrix}
        0   \\
        0   \\
        V_1 \\
        V_2
    \end{bmatrix}
\end{equation}

The matrices $A$, $B$ and $C$ are then defined as:

\begin{equation}
    A = \begin{bmatrix}
        ? & ? & ? & ? \\
        ? & ? & ? & ? \\
        ? & ? & ? & ? \\
        ? & ? & ? & ?
    \end{bmatrix}
    \quad
    B = \begin{bmatrix}
        0 \\
        0 \\
        ? \\
        ?
    \end{bmatrix}
    \quad
    C = \begin{bmatrix}
        1 & 0 & 0 & 0
    \end{bmatrix}
\end{equation}
