\subsection{State Space Representation}
\label{subsec:state_space_representation}

\hl{We need to write how the operating point vector has been computed}

In the optics of control theory, it is useful to represent the system in the state space form.
The state space representation is a mathematical model of a physical system as a set of input, output and state variables related by first-order differential equations.
The state space representation is particularly useful for linear systems, as it allows to easily apply linear control techniques.

A generic nonlinear system can be represented in the state space form as:

\begin{equation}
    \begin{aligned}
        \dot{\mathbf{x}} & = f(\mathbf{x}, \mathbf{u}) \\
        \mathbf{y}       & = g(\mathbf{x}, \mathbf{u})
    \end{aligned}
\end{equation}

Where $\mathbf{x}$ is the state vector and $\mathbf{u}$ is the input vector.

Similarly to what has been done in the previous subsection, we can perform a linearization of the system around an operating point to obtain the linearized state space representation in the form of:

\begin{equation}
    \begin{aligned}
        \dot{\delta\mathbf{x}} & \approx A \delta\mathbf{x} + B \delta\mathbf{u} \\
        \mathbf{y}             & \approx C \delta\mathbf{x} + D \delta\mathbf{u}
    \end{aligned}
\end{equation}

Where $\delta\mathbf{x}$ and $\delta\mathbf{u}$ are the deviations of the state and input vectors from the operating point, respectively.
While the matrices $A$, $B$, $C$ and $D$ are the Jacobian matrices with respect to the state and input vectors evaluated at the operating point.

\paragraph{MagLev System State Space Representation}

Given the linearized model derived in the previous subsection, we can define the state vector $\mathbf{x}$ and the input vector $\mathbf{u}$ as:

\begin{equation}
    \mathbf{x} = \begin{bmatrix}
        z   \\
        v   \\
        I_1 \\
        I_2
    \end{bmatrix}
    \quad
    \mathbf{u} = \begin{bmatrix}
        U_1 \\
        U_2
    \end{bmatrix}
\end{equation}

Once the state and input vectors have been defined, the linearized state space representation can be obtained by leveraging the linearized model derived in the previous subsection.
The matrices $A$, $B$, $C$ and $D$ are then defined as:

\begin{equation}
    \begin{aligned}
        A & = \frac{\partial f}{\partial \mathbf{x}} \Bigg|_{(\mathbf{x}_{op}, \mathbf{u}_{op})}
        = \begin{bmatrix}
              0      & 1 & 0      & 0      \\
              a_{21} & 0 & a_{23} & a_{24} \\
              a_{31} & 0 & a_{33} & 0      \\
              a_{41} & 0 & 0      & a_{44}
          \end{bmatrix}                                                           \\
        B & = \frac{\partial f}{\partial \mathbf{u}} \Bigg|_{(\mathbf{x}_{op}, \mathbf{u}_{op})}
        = \begin{bmatrix}
              0      & 0      \\
              0      & 0      \\
              b_{31} & 0      \\
              0      & b_{42}
          \end{bmatrix}                                                                        \\
        C & = \frac{\partial g}{\partial \mathbf{x}} \Bigg|_{(\mathbf{x}_{op}, \mathbf{u}_{op})}
        = \begin{bmatrix}
              1 & 0 & 0 & 0
          \end{bmatrix}                                                                         \\
        D & = \frac{\partial g}{\partial \mathbf{u}} \Bigg|_{(\mathbf{x}_{op}, \mathbf{u}_{op})}
        = \begin{bmatrix}
              0 & 0
          \end{bmatrix}
    \end{aligned}
\end{equation}

By leveraging the linearization of the model derived in the previous subsection and reported in Equation \ref{eq:linearized_model}, the elements of the matrices $A$, $B$, $C$ and $D$ can be computed as:

\begin{equation}
    \begin{aligned}
        a_{21} & = \frac{1}{m} \left( \frac{1}{2} \frac{\partial^2 L_1}{\partial z^2} I_1^2 + \frac{1}{2} \frac{\partial^2 L_2}{\partial z^2} I_{2}^2 \right) \bigg|_{(\mathbf{x}_{op}, \mathbf{u}_{op})} \\
        a_{23} & = \frac{1}{m} \left( \frac{\partial L_1}{\partial z} I_1\right) \bigg|_{(\mathbf{x}_{op}, \mathbf{u}_{op})}                                                                              \\
        a_{24} & = \frac{1}{m} \left( \frac{\partial L_2}{\partial z} I_2\right) \bigg|_{(\mathbf{x}_{op}, \mathbf{u}_{op})}                                                                              \\
        a_{31} & = \left(- L_1^{-2} \frac{\partial L_1}{\partial z} \left(- R_1 I_1 + k_1 U_1 + c_1 \right) \right) \bigg|_{(\mathbf{x}_{op}, \mathbf{u}_{op})}                                           \\
        a_{33} & = \left(- L_1^{-1} R_1 \right) \big|_{(\mathbf{x}_{op}, \mathbf{u}_{op})}                                                                                                                \\
        a_{41} & = \left(- L_2^{-2} \frac{\partial L_2}{\partial z} \left(- R_2 I_2 + k_2 U_2 + c_2 \right) \right) \bigg|_{(\mathbf{x}_{op}, \mathbf{u}_{op})}                                           \\
        a_{44} & = \left(- L_2^{-1} R_2 \right) \big|_{(\mathbf{x}_{op}, \mathbf{u}_{op})}                                                                                                                \\
        b_{31} & = \left(L_1^{-1} k_1 \right) \big|_{(\mathbf{x}_{op}, \mathbf{u}_{op})}                                                                                                                  \\
        b_{42} & = \left(L_2^{-1} k_2 \right) \big|_{(\mathbf{x}_{op}, \mathbf{u}_{op})}
    \end{aligned}
\end{equation}


