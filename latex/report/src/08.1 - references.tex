\subsection{References}
\label{subsec:references}

The step function is a classical and widely used method for evaluating the performance of controllers around the linearization point, as it provides a clear and interpretable measure of the system's transient and steady-state response. In this case, the system is linearized at a position of 0.010 m from the upper coil. The step response is initiated after 2 seconds, moving the sphere to 0.012 m. This setup was chosen because it allows for the evaluation of the controller's ability to handle a straightforward set-point change, focusing on the system's stability, overshoot, settling time, and steady-state error around the linearization point.

The multistep test extends the evaluation by moving the sphere in both directions relative to the linearization point, with steps of 0.012 m and 0.008 m, each of magnitude 0.002 m. This approach ensures that the controller's performance is assessed not only for a single positive deviation but also for negative deviations. It also checks how the controller manages transitions between closer and farther positions from the upper coil.

The stairs reference use a smaller step value of 0.001 m, creating multiple sequential movements of the sphere. This finer resolution enables a detailed observation of the system's behavior when it operates slightly further from the linearization point. This is particularly useful for understanding how the controller deals with minor perturbations and gradual positional adjustments.

Finally, the sinusoidal references both fast and slow sines introduce continuous oscillatory motion between 0.012 m and 0.008 m. By varying the time periods to 2 seconds and 6 seconds, respectively, the tests analyze the controller's response to different rates of change. The fast sine evaluates the system's ability to track rapid oscillations, highlighting its bandwidth and response speed, while the slow sine provides insights into the performance during more gradual, steady oscillations. Together, these tests offer a comprehensive view of the controller's dynamic capabilities and its ability to adapt to varying rates of input changes, providing valuable information about its flexibility and robustness.
