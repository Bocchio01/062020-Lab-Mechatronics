\subsection{Model derivation}
\label{subsec:model_derivation}

The \acrshort{mls} is a complex system that can be divided into two main subsystems:

\begin{itemize}
    \item \textbf{Electromagnetic subsystem}: it takes into account the electrical components of the system from the power supply to the generation of the magnetic field by the coils;
    \item \textbf{Mechanical subsystem}: it takes into account the dynamics of the ball and all the forces acting on it, including the electromagnetic forces generated by the magnetic field.
\end{itemize}

Due to the presence of the ball that moves inside a magnetic field, a complex connection between the two subsystems that goes beyond the simple force balance exists.
For this reason, it's almost impossible to derive a complete model without considering both subsystems at the same time.

In Figure \ref{fig:system_model}, a schematic representation of all the components and forces acting on the system is shown.
Instead, in Table \ref{tab:components} a brief description of the components of the system and their units is reported.

\begin{figure}[H]

    \begin{minipage}{0.40\textwidth}

        \centering

        \begin{tikzpicture}[european voltages]

            \def\radius{0.3}

            % Upper circuit
            \node at (-1.0, 3.5) {$T_1$};
            \draw (-3, 3.5) node [right] {$+$}
            to [short] ++(0, -1)
            to [R, l^=$R_1$, resistors/zigs=6] ++(2, 0)
            to [variable cute inductor, i>^=$I_1$, l=$L_1$] ++(2, 0)
            to [short] ++(0, +1) node [right] {$-$};

            % Reference system
            \draw[|->] (-1.0, +2.5) -- ++(0, -3) node[left] {$z$, $\dot{z}$, $\ddot{z}$};

            % Ball
            \filldraw[fill=gray, draw=black] (0, 0) circle (\radius);

            % Upward forces
            \draw[thick, ->] (-0.1, +\radius) -- ++(0, +1.5) node[right] {$F_{\text{em1}}$};
            \draw[thick, ->] (+0.0, +\radius) -- ++(0, +1.0) node[right] {$F_{\text{in}}$};
            \draw[thick, ->] (+0.1, +\radius) -- ++(0, +0.5) node[right] {$F_{\text{d}}$};

            % Downward forces
            \draw[thick, ->] (+0.1, -\radius) -- ++(0, -0.5) node[right] {$F_{\text{g}}$};
            \draw[thick, ->] (-0.1, -\radius) -- ++(0, -1.0) node[right] {$F_{\text{em2}}$};

            % Lower circuit
            \node at (-1.0, -3.0) {$T_2$};
            \draw (-3, -3) node [right] {$+$}
            to [short] ++(0, +1)
            to [R, l_=$R_2$, resistors/zigs=6] ++(2, 0)
            to [variable cute inductor, i>_=$I_2$, l_=$L_2$] ++(2, 0)
            to [short] ++(0, -1) node [right] {$-$};

        \end{tikzpicture}

    \end{minipage}
    %
    \hfill
    %
    \begin{minipage}{0.55\textwidth}

        \centering

        \begin{tabular}{|c|l|c|}
            \hline
            \textbf{Name}      & \textbf{Description}             & \textbf{Units} \\
            \hline
            $F_{\text{g}}$     & Gravitational force              & N              \\
            $F_{\text{in}}$    & Inertial force                   & N              \\
            $F_{\text{d}}$     & Drag force                       & N              \\
            $F_{\text{em1,2}}$ & Electromagnetic force            & N              \\
            \hline
            $R_{1,2}$          & Resistance of the coil           & $\Omega$       \\
            $L_{1,2}$          & Inductance of the coil           & H              \\
            $I_{1,2}$          & Current flowing through the coil & A              \\
            $V_{1,2}$          & Voltage applied to the coil      & V              \\
            $T_{1,2}$          & Temperature of the coil          & $^\circ C$     \\
            \hline
        \end{tabular}

    \end{minipage}

    \caption{Schematic representation of the \acrshort{mls} system and description of the components.}
    \label{fig:system_model}
    \label{tab:components}

\end{figure}


In the following sections, we will derive the equations that governs the \acrshort{mls} system, adopting an energetic approach that starts from the energy conservation principle.

\subsubsection{Mathematical model}
\label{subsubsec:mathematical_model}

We can now proceed with the derivation of the equations that govern the system.

At first, we can recall that the energy conservation principle states that the sum of the kinetic, potential, and dissipated energy of the system is equivalent to the work done by the external forces acting on the system.

\paragraph{Lagrange's equation}

Thanks to the Lagrange's equation we write the following, that encaptulates the energy conservation principle:

\begin{equation}
    \frac{d}{dt} \left( \frac{\partial \mathcal{T}}{\partial \dot{\mathbf{u}}} \right) - \frac{\partial \mathcal{T}}{\partial \mathbf{u}} + \frac{\partial \mathcal{D}}{\partial \dot{\mathbf{u}}} + \frac{\partial \mathcal{U}}{\partial \mathbf{u}} = \mathcal{Q}
    \label{eq:lagrange_equation}
\end{equation}

Where $\mathbf{u}$ is the generalized coordinates of the system, $\mathbf{T}$ is the kinetic energy of the system, $\mathbf{D}$ is the dissipated energy of the system, $\mathbf{U}$ is the potential energy of the system, and $\mathbf{Q}$ is the generalized input to the system.

At first, we can give a definition of all the energetic terms included in Equation \ref{eq:lagrange_equation} for the \acrshort{mls} system.
Notice that with respect to traditional purely mechanical systems, we also have to consider the stored energy in the coils as inductors, the dissipation due to the resistance of the coils, and the potential energy given by the external power supply.

By doing so, we can write the kinetic energy of the system as:

\begin{equation}
    \mathcal{T} = \frac{1}{2} m \dot{z}^2 + \frac{1}{2} L_1(z, \dot{q_1}, T_1) \dot{q_1}^2 + \frac{1}{2} L_2(z, \dot{q_2}, T_2) \dot{q_2}^2
    \label{eq:kinetic_energy}
\end{equation}

Where $m$ is the mass of the ball, $L_1$ and $L_2$ are the inductances of the coils, and $q_1$ and $q_2$ are the charges stored in the coils.
It follows that $\dot{q_1}$ and $\dot{q_2}$ are the currents flowing through the coils.

The dissipated energy of the system can be written as:

\begin{equation}
    \mathcal{D} = \int_{\dot{z}(\cdot)} \frac{1}{2} C_d A \rho \dot{z}^2 d\dot{z} + \int_{\dot{q_1}(\cdot)} R_1(\dot{q_1}, T_1) \dot{q_1} d\dot{q_1} + \int_{\dot{q_2}(\cdot)} R_2(\dot{q_2}, T_2) \dot{q_2} d\dot{q_2}
    \label{eq:dissipated_energy}
\end{equation}

Where $C_d$ is the drag coefficient, $A$ is the cross-sectional area of the ball, and $\rho$ is the density of the air.

Instead, the potential energy of the system can be written as:

\begin{equation}
    \mathcal{U} = -m g z - q_1 V_1 - q_2 V_2
    \label{eq:potential_energy}
\end{equation}

Where $V_1$ and $V_2$ are the voltages applied to the coils.

Finally, the generalized input to the system can be evaluated as:

\begin{equation}
    \mathcal{Q} = 0
    \label{eq:generalized_input}
\end{equation}

For convenience, we have chosen to consider both the external power supplied to the coils and the gravitational force as potential energy terms and not as generalized inputs.
Notice also the minus sign in the gravitational potential energy term, which is due to the fact that the gravitational force increases the potential energy with respect to the chosen reference frame (positive downwards).

\paragraph{Electrical components model}

Before proceeding, it's necessary to explicitly state the dependence of the inductance and resistance terms on the generalized coordinates of the system.

We can assume that, in first approximation, the sensitivity of both the electrical components to the temperature of the coils is negligible.
This is strong and possibly incorrect assumption, but it allows us to simplify the model and focus on the main dynamics of the system.

For what regards the resistance terms, we can assume that the resistance of the coils is constant and so does not depend on neither the current flowing through them nor the temperature of the coils.
For the assumption stated above, we can write the resistance terms as:

\begin{equation}
    \begin{aligned}
        R_1 & = R_1(\dot{q_1}, T_1) = R_{10} \\
        R_2 & = R_2(\dot{q_2}, T_2) = R_{20}
    \end{aligned}
    \label{eq:model_for_resistance}
\end{equation}

Where $R_{*0}$ are the resistances of the coils at ambient temperature with negligible current flowing through them.

Instead, considering the inductance terms, we will neglect the dependence on the coil's temperature, but we will take into account the variation of the inductance due to the presence of the ball in the magnetic field (principal source of nonlinearity in the system) and also the dependence over the current flowing through the coils.
For the assumption stated above, we will try to model the inductance terms as:

\begin{equation}
    \begin{aligned}
        L_1 & = L_1(z, \dot{q_1}, T_1) = L_{10} + L_{1z} e^{-a_1 z} + L_{1I} * \tanh(-b_1 * I_{1})            \\
        L_2 & = L_2(z, \dot{q_2}, T_2) = L_{20} + L_{2z} e^{-a_2 (h - 2r - z)} + L_{2I} * \tanh(-b_2 * I_{2})
    \end{aligned}
    \label{eq:model_for_inductance}
\end{equation}

Where $L_{*0}$ are the nominal inductances values. Instead, $L_{*z}$, $a_*$ and $L_{*I}$, $b_*$ are coefficients that take into account the variation of the inductance due to the presence of the ball in the magnetic field and the current flowing through the coils, respectively.

It has to be noted that this model was suggested by a careful analysis of experimental data and is not directly based on theoretical considerations.
Some previous models of inductance can also be found in the literature, but they are often too complex and not suitable for control purposes.

\paragraph{Equations of motion}

To derive the equations of motion of the system, we can substitute the kinetic (\ref{eq:kinetic_energy}), dissipated (\ref{eq:dissipated_energy}), and potential energy (\ref{eq:potential_energy}) terms into the Lagrange's equation (\ref{eq:lagrange_equation}).
By a qualitative analysis of the system, recalling also that we have chosen to neglect the effect of coil's temperature for the electrical components, we can see that the generalized coordinates are $z$, $q_1$, and $q_2$, and so the vector of generalized coordinates is $\mathbf{u} = [z, q_1, q_2]^T$.

Once $\mathbf{u}$ has been identified, the procedure to derive the equations of motion is straightforward.
Following Equation \ref{eq:lagrange_equation}, we can write the following system of equations:

\begin{equation}
    \begin{cases}
        \frac{d}{dt} \left( \frac{\partial \mathcal{T}}{\partial \dot{z}} \right) - \frac{\partial \mathcal{T}}{\partial z} + \frac{\partial \mathcal{D}}{\partial \dot{z}} + \frac{\partial \mathcal{U}}{\partial z} = \mathcal{Q}         \\
        \frac{d}{dt} \left( \frac{\partial \mathcal{T}}{\partial \dot{q_1}} \right) - \frac{\partial \mathcal{T}}{\partial q_1} + \frac{\partial \mathcal{D}}{\partial \dot{q_1}} + \frac{\partial \mathcal{U}}{\partial q_1} = \mathcal{Q} \\
        \frac{d}{dt} \left( \frac{\partial \mathcal{T}}{\partial \dot{q_2}} \right) - \frac{\partial \mathcal{T}}{\partial q_2} + \frac{\partial \mathcal{D}}{\partial \dot{q_2}} + \frac{\partial \mathcal{U}}{\partial q_2} = \mathcal{Q}
    \end{cases}
\end{equation}

By substituting the energetic terms obtained in Equations \ref{eq:kinetic_energy}, \ref{eq:dissipated_energy}, \ref{eq:potential_energy}, \ref{eq:generalized_input} into the set of equations above, we obtain the following equations of motion:

\begin{equation}
    \begin{cases}
        m \ddot{z} - \frac{1}{2} \frac{\partial L_1}{\partial z} \dot{q_1}^2 - \frac{1}{2} \frac{\partial L_2}{\partial z} \dot{q_2}^2 + \frac{1}{2} C_d A \rho \dot{z} |\dot{z}| - m g = 0                                                                                                                                                                                                            \\
        \frac{1}{2} \left( \frac{\partial^2 L_1}{\partial \dot{q_1} \partial z} \dot{z} + \frac{\partial^2 L_1}{\partial \dot{q_1}^2} \ddot{q_1} \right) \dot{q_1}^2 + \frac{\partial L_1}{\partial \dot{q_1}} \dot{q_1} \ddot{q_1} + \left( \frac{\partial L_1}{\partial z} \dot{z} + \frac{\partial L_1}{\partial \dot{q_1}} \ddot{q_1} \right) \dot{q_1} + L_1 \ddot{q_1} + R_1 \dot{q_1} - V_1 = 0 \\
        \frac{1}{2} \left( \frac{\partial^2 L_2}{\partial \dot{q_2} \partial z} \dot{z} + \frac{\partial^2 L_2}{\partial \dot{q_2}^2} \ddot{q_2} \right) \dot{q_2}^2 + \frac{\partial L_2}{\partial \dot{q_2}} \dot{q_2} \ddot{q_2} + \left( \frac{\partial L_2}{\partial z} \dot{z} + \frac{\partial L_2}{\partial \dot{q_2}} \ddot{q_2} \right) \dot{q_2} + L_2 \ddot{q_2} + R_2 \dot{q_2} - V_2 = 0 \\
    \end{cases}
\end{equation}

For convenience, we can replace time derivatives of charges with currents by using the definition of current as the time derivative of the charge.
Moreover, we can group the terms in the equations above so to move higher order derivatives to the left-hand side of the equations.
Finally, we also transform the second order differential equations into first order differential equations by introducing a fourth equation and considering the ball velocity $v$ as a state variable.
By doing so, we obtain the following set of equations:

\begin{equation}
    \begin{cases}
        \dot{z} = v                                                                                                                                                                                                                                                                                       \\
        \dot{v} = m^{-1} \left(\frac{1}{2} \frac{\partial L_1}{\partial z} I_1^2 + \frac{1}{2} \frac{\partial L_2}{\partial z} I_2^2 - \frac{1}{2} C_d A \rho \dot{z} |\dot{z}| + m g  \right)                                                                                                            \\

        \dot{I_1} = \left( \frac{1}{2} \frac{\partial^2 L_1}{\partial I_1^2} I_1^2 + 2\frac{\partial L_1}{\partial I_1} I_1 + L_1 \right)^{-1} \left( -\frac{1}{2} \frac{\partial^2 L_1}{\partial I_1 \partial z} \dot{z} I_1^2 - \frac{\partial L_1}{\partial z} \dot{z} I_1 - R_1 I_1 + V_1 \right) = 0 \\
        \dot{I_2} = \left( \frac{1}{2} \frac{\partial^2 L_2}{\partial I_2^2} I_2^2 + 2\frac{\partial L_2}{\partial I_2} I_2 + L_2 \right)^{-1} \left( -\frac{1}{2} \frac{\partial^2 L_2}{\partial I_2 \partial z} \dot{z} I_2^2 - \frac{\partial L_2}{\partial z} \dot{z} I_2 - R_2 I_2 + V_2 \right) = 0
    \end{cases}
    \label{eq:equations_of_motion_full}
\end{equation}

The set of equations above represents the complete mathematical model of the \acrshort{mls} system.
One can notice that the equations are both nonlinear and coupled, making the system hard to analyze and control.

\paragraph{Model reduction}

In order to simplify the model and make it more suitable for control purposes, we can make some assumptions that allow us to reduce the complexity of the system.

Based also on the experimental data collected during the parameters' identification phase (Subsection \ref{subsec:parameters_identification}), we can state that the sensitivity of the inductance terms to the current flowing through the coils is negligible around the operating point.
Moreover, also the velocity of the ball will always be small, and so every term that is linearly dependent on it can be neglected.
By doing so, we can impose the following conditions to the system:

\begin{equation}
    \begin{cases}
        \frac{\partial L_*}{\partial I_*}     & \approx 0 \\
        \frac{\partial^2 L_*}{\partial I_*^2} & \approx 0 \\
        \dot{z}                               & \approx 0
    \end{cases}
    \label{eq:model_reduction_conditions}
\end{equation}

By doing so, we can simplify the equations of motion as:

\begin{equation}
    \begin{cases}
        \dot{z} = v                                                                                                                                 \\
        \dot{v} = m^{-1} \left(\frac{1}{2} \frac{\partial L_1}{\partial z} I_1^2 + \frac{1}{2} \frac{\partial L_2}{\partial z} I_2^2 + m g  \right) \\
        \dot{I_1} = L_1^{-1} \left(- R_1 I_1 + V_1 \right)                                                                                          \\
        \dot{I_2} = L_2^{-1} \left(- R_2 I_2 + V_2 \right)
    \end{cases}
    \label{eq:reduced_equations_of_motion}
\end{equation}

\paragraph{Control input adaptation}

A final important remark has to be made about the so-called \textit{black zone} of the system, that are the regions where the current flowing through the coils is no more reachable.

In particular, by simply connect the power supply to the coils, a minimum voltage will be applied and a certain amount of current will flow through the coils.
In the following, we will refer to this current and voltage as $I_{*min}$ and $V_{*min}$ respectively.

Because of this, the above derived model must be slightly modified to take into account the black zone of the system.

In particular, the set of Equations \ref{eq:reduced_equations_of_motion} can be rewritten as:

\begin{equation}
    \begin{cases}
        \dot{z} = v                                                                                                                                 \\
        \dot{v} = m^{-1} \left(\frac{1}{2} \frac{\partial L_1}{\partial z} I_1^2 + \frac{1}{2} \frac{\partial L_2}{\partial z} I_2^2 + m g  \right) \\
        \dot{I_1} = L_1^{-1} \left(- R_1 I_1 + (k_1 U_1 + c_1) \right)                                                                              \\
        \dot{I_2} = L_2^{-1} \left(- R_2 I_2 + (k_2 U_2 + c_2) \right)
    \end{cases}
    \label{eq:reduced_equations_of_motion_final}
\end{equation}


\subsubsection{Literature model}
\label{subsubsec:literature_model}

In the literature, the model of the \acrshort{mls} system is often further simplified by considering empirical values associated with the inductances and resistances of the coils.
In particular, from the \texttt{Inteco} manual, the following set of equations is reported:

\begin{equation}
    \begin{cases}
        \dot{z} = v                                                  \\
        \dot{v} = m^{-1} \left(-F_{em1} + F_{em2} + m g  \right)     \\
        \dot{I_1} = \frac{1}{f(z)} \left(- I_1 + ki U_1 + ci \right) \\
        \dot{I_2} = \frac{1}{f(h - 2r - z)} \left(- I_2 + ki U_2 + ci \right)
    \end{cases}
    \label{eq:simplified_equations_of_motion_final}
\end{equation}

Where $f(x)$ is an empirical function that takes into account the variation of the inductances due to the presence of the ball in the magnetic field and has the following form:

\begin{equation}
    f(z) = \frac{f_{IP1}}{f_{IP2}} e^{\left(-\frac{z}{f_{IP2}}\right)}
\end{equation}

While $F_{em1}$ and $F_{em2}$ are the electromagnetic forces acting on the ball and have the following form:

\begin{equation}
    \begin{cases}
        F_{em1} & = \frac{F_{emP1}}{F_{emP2}} e^{-\frac{z}{F_{emP2}}}  I_1^2          \\
        F_{em2} & = \frac{F_{emP1}}{F_{emP2}} e^{-\frac{h - 2r - z}{F_{emP2}}}  I_2^2
    \end{cases}
\end{equation}
