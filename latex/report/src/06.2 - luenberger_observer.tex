\subsection{Luenberger Observer}
\label{sec:luenberger_observer}

The Luenberger observer is a state observer that allows to estimate the state of a system, given the input coming from the controllers and at least one measured output.
The observer is designed in such a way that the error between the estimated state and the real state converges to zero, as time goes to infinity.

To do so, one can consider the following dynamical system:

\begin{equation}
    \begin{cases}
        \dot{\hat{x}} = A \hat{x} + B u + L(y - C \hat{x}) \\
        \hat{y} = C \hat{x}
    \end{cases}
    \label{eq:observer_dynamics}
\end{equation}

Where $\hat{x}$ is the estimated state, $\hat{y}$ is the estimated output, $L$ is the observer gain, and $y$ is the measured output of the system.

The poles of the observer are given by the eigenvalues of the matrix $A - LC$, and the observer is stable if the poles are placed in the left half plane of the complex plane.

The observer gain $L$ can be computed using the Ackermann formula, which is a generalization of the pole placement method for state-space systems.

\paragraph{Design}

Given that there are no restrictions (except for being in the left-hand side of the complex plane) for the position of the poles, we choose by chance the followings:

\begin{equation}
    eig(A-LC) = \begin{bmatrix}
        ?? \\
        ?? \\
        ??
    \end{bmatrix}
    \label{eq:luenberger_observer_poles}
\end{equation}
