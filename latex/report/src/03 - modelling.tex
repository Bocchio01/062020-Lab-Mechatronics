\section{Modelling}
\label{sec:modelling}

The \acrshort{mls} is a complex system that can be divided into:

\begin{itemize}
    \item \textbf{Electromagnetic subsystem}: it takes into account all the electrical components going from the power supply to the electromagnets themselves, and so the generation of the magnetic field by the coils;
    \item \textbf{Mechanical subsystem}: it takes into account the dynamics of the ball and the forces acting on it, including the electromagnetic forces generated by the magnetic field.
\end{itemize}

Due to the presence of the ball that moves inside a magnetic field, a complex connection between the two subsystems that goes beyond the simple force balance exists.
For this reason, it's almost impossible to derive a complete model without considering both subsystems at the same time.

In Figure \ref{fig:system_model}, a schematic representation of all the components of the system and the forces acting on it is shown.
Instead, in Table \ref{tab:components} a brief description of the components is provided.

\begin{figure}[H]

    \begin{minipage}{0.40\textwidth}

        \centering

        \begin{tikzpicture}[european voltages]

            \def\radius{0.3}

            % Upper circuit
            \node at (-1.0, 3.5) {$T_1$};
            \draw (-3, 3.5) node [right] {$+$}
            to [short] ++(0, -1)
            to [R, l^=$R_1$, resistors/zigs=6] ++(2, 0)
            to [variable cute inductor, i>^=$I_1$, l=$L_1$] ++(2, 0)
            to [short] ++(0, +1) node [right] {$-$};

            % Reference system
            \draw[|->] (-1.0, +2.5) -- ++(0, -3) node[left] {$z$, $\dot{z}$, $\ddot{z}$};

            % Ball
            \filldraw[fill=gray, draw=black] (0, 0) circle (\radius);

            % Upward forces
            \draw[thick, ->] (-0.1, +\radius) -- ++(0, +1.5) node[right] {$F_{\text{em1}}$};
            \draw[thick, ->] (+0.0, +\radius) -- ++(0, +1.0) node[right] {$F_{\text{in}}$};
            \draw[thick, ->] (+0.1, +\radius) -- ++(0, +0.5) node[right] {$F_{\text{d}}$};

            % Downward forces
            \draw[thick, ->] (+0.1, -\radius) -- ++(0, -0.5) node[right] {$F_{\text{g}}$};
            \draw[thick, ->] (-0.1, -\radius) -- ++(0, -1.0) node[right] {$F_{\text{em2}}$};

            % Lower circuit
            \node at (-1.0, -3.0) {$T_2$};
            \draw (-3, -3) node [right] {$+$}
            to [short] ++(0, +1)
            to [R, l_=$R_2$, resistors/zigs=6] ++(2, 0)
            to [variable cute inductor, i>_=$I_2$, l_=$L_2$] ++(2, 0)
            to [short] ++(0, -1) node [right] {$-$};

        \end{tikzpicture}

    \end{minipage}
    %
    \hfill
    %
    \begin{minipage}{0.55\textwidth}

        \centering

        \begin{tabular}{|c|l|c|}
            \hline
            \textbf{Name}      & \textbf{Description}               & \textbf{Units} \\
            \hline
            $F_{\text{g}}$     & Gravitational force                & N              \\
            $F_{\text{in}}$    & Inertial force                     & N              \\
            $F_{\text{d}}$     & Drag force                         & N              \\
            $F_{\text{em1,2}}$ & Electromagnetic forces             & N              \\
            \hline
            $R_{1,2}$          & Resistances of the coils           & $\Omega$       \\
            $L_{1,2}$          & Inductances of the coils           & H              \\
            $I_{1,2}$          & Currents flowing through the coils & A              \\
            $V_{1,2}$          & Voltages applied to the coils      & V              \\
            $T_{1,2}$          & Temperatures of the coils          & $^\circ C$     \\
            \hline
        \end{tabular}

    \end{minipage}

    \caption{Schematic representation of the \acrshort{mls} system and description of its components.}
    \label{fig:system_model}
    \label{tab:components}

\end{figure}


In the following sections, we will derive the equations that governs the \acrshort{mls} system, adopting an energetic approach that starts from the energy conservation principle.

\subsection{Mathematical model}
\label{subsec:mathematical_model}

We can now proceed with the derivation of the equations that govern the system.

At first, we can recall the energy conservation principle stating that the sum of the kinetic, potential, and dissipated energy of the system is equivalent to the work done by the external forces acting on it.



\subsubsection{Lagrangian formulation}
\label{subsubsec:lagrangian_formulation}

Thanks to Lagrange's equation we encapsulate the energy conservation principle by writing the following:

\begin{equation}
    \frac{d}{dt} \left( \frac{\partial \mathcal{T}}{\partial \dot{\mathbf{u}}} \right) - \frac{\partial \mathcal{T}}{\partial \mathbf{u}} + \frac{\partial \mathcal{D}}{\partial \dot{\mathbf{u}}} + \frac{\partial \mathcal{U}}{\partial \mathbf{u}} = \mathcal{Q}
    \label{eq:lagrange_equation}
\end{equation}

Where $\mathbf{u}$ is the generalized coordinates of the system, $\mathbf{T}$ is the kinetic energy, $\mathbf{D}$ is the dissipated one, $\mathbf{U}$ is the potential energy, and $\mathbf{Q}$ is the generalized input to the system.

At first, we can give a definition of all the energetic terms included in Equation \ref{eq:lagrange_equation} for the \acrshort{mls} system.
Notice that with respect to traditional purely mechanical systems, we also have to consider the stored energy in the coils as inductors, the dissipation due to the resistance of the coils, and the potential energy given by the external power supply.

By doing so, we can write the kinetic energy of the system as:

\begin{equation}
    \mathcal{T} = \frac{1}{2} m \dot{z}^2 + \frac{1}{2} L_1(z, \dot{q_1}, T_1) \dot{q_1}^2 + \frac{1}{2} L_2(z, \dot{q_2}, T_2) \dot{q_2}^2
    \label{eq:kinetic_energy}
\end{equation}

Where $m$ is the mass of the ball, $L_1$ and $L_2$ are the inductances of the coils, and $q_1$ and $q_2$ are the charges stored in the coils.
It follows that $\dot{q_1}$ and $\dot{q_2}$ are the currents flowing through the coils.

The dissipated energy of the system can be written as:

\begin{equation}
    \mathcal{D} = \int_{\dot{z}(\cdot)} \frac{1}{2} C_d A \rho \dot{z}^2 d\dot{z} + \int_{\dot{q_1}(\cdot)} R_1(\dot{q_1}, T_1) \dot{q_1} d\dot{q_1} + \int_{\dot{q_2}(\cdot)} R_2(\dot{q_2}, T_2) \dot{q_2} d\dot{q_2}
    \label{eq:dissipated_energy}
\end{equation}

Where $C_d$ is the drag coefficient for a generic spherical object, $A$ is the cross-sectional area of the ball, and $\rho$ is the density of air.

Instead, the potential energy of the system can be written as:

\begin{equation}
    \mathcal{U} = -m g z - q_1 V_1 - q_2 V_2
    \label{eq:potential_energy}
\end{equation}

Where $V_1$ and $V_2$ are the voltages applied to the coils.

Finally, the generalized input to the system can be evaluated as:

\begin{equation}
    \mathcal{Q} = 0
    \label{eq:generalized_input}
\end{equation}

For convenience, we have chosen to consider both the external power supplied and the gravitational force as potential energy terms and not as generalized inputs.
Notice also the minus sign in the gravitational potential energy term, which is due to the fact that the gravitational force tends to increase the potential energy with respect to the chosen reference frame (positive downwards, see Figure \ref{fig:system_model}).



\subsubsection{Electrical components model}
\label{subsubsec:electrical_components_model}

Before proceeding, it's necessary to explicitly state the dependence of the inductance and resistance terms on the generalized coordinates of the system.

We can assume that, in first approximation, the sensitivity of both the electrical components to the temperature of the coils is negligible.
This is strong and possibly incorrect assumption, but it allows us to simplify the model and focus on the main dynamics of the system.

\paragraph{Resistance model}

For what regards the resistance terms, we can assume that the resistance of the coils is constant and so does not depend on neither the current flowing through them nor the temperature of the coils.
Under these assumptions, we can write the resistance terms as:

\begin{equation}
    \begin{aligned}
        R_1 & = R_1(\dot{q_1}, T_1) = R_{10} \\
        R_2 & = R_2(\dot{q_2}, T_2) = R_{20}
    \end{aligned}
    \label{eq:model_for_resistance}
\end{equation}

Where $R_{*0}$ are the resistances of the coils measured at ambient temperature with negligible current flowing through them.

\paragraph{Inductance model}

Considering the inductance terms, we will again neglect the dependence on the coil's temperature, but we will take into account the variation of the inductance due to the presence of the ball in the magnetic field (principal source of nonlinearity in the system) and also the dependence over the current flowing through the coils.
For the assumption stated above, we will model the inductance terms as:

\begin{equation}
    \begin{aligned}
        L_1 & = L_1(z, \dot{q_1}, T_1) = L_{10} + L_{1z} e^{-a_{1z} z} + L_{1I} * \arctan(a_{1I} I_{1} - b_{1I})            \\
        L_2 & = L_2(z, \dot{q_2}, T_2) = L_{20} + L_{2z} e^{-a_{2z} (h - 2r - z)} + L_{2I} * \arctan(a_{2I} I_{2} - b_{2I})
    \end{aligned}
    \label{eq:model_for_inductance}
\end{equation}

Where $L_{*0}$ are the nominal inductances values.
Instead, $L_{*z}$, $a_{*z}$ and $L_{*I}$, $a_{*I}$, $b_{*I}$ are coefficients that take into account the variation of the inductance due to the presence of the ball in the magnetic field and the current flowing through the coils, respectively.

It has to be noted that this model was suggested by a careful analysis of experimental data and is not directly based on theoretical considerations.
Some previous models of inductance can also be found in the literature, but they are often too complex and not suitable for control purposes.



\subsubsection{Equations of motion}
\label{subsubsec:equations_of_motion}

Considering the assumptions made in both the resistance and inductance models (Equations \ref{eq:model_for_resistance}, \ref{eq:model_for_inductance}) to neglect their dependence on the temperature of the coils, we can see that the generalized coordinates are $z$, $q_1$, and $q_2$, and so the vector of generalized coordinates is $\mathbf{u} = [z, q_1, q_2]^T$.

Once $\mathbf{u}$ has been identified, the procedure to derive the equations of motion is straightforward.
Based on Equation \ref{eq:lagrange_equation}, we can write the following system of equations:

\begin{equation}
    \begin{cases}
        \frac{d}{dt} \left( \frac{\partial \mathcal{T}}{\partial \dot{z}} \right) - \frac{\partial \mathcal{T}}{\partial z} + \frac{\partial \mathcal{D}}{\partial \dot{z}} + \frac{\partial \mathcal{U}}{\partial z} = \mathcal{Q}         \\
        \frac{d}{dt} \left( \frac{\partial \mathcal{T}}{\partial \dot{q_1}} \right) - \frac{\partial \mathcal{T}}{\partial q_1} + \frac{\partial \mathcal{D}}{\partial \dot{q_1}} + \frac{\partial \mathcal{U}}{\partial q_1} = \mathcal{Q} \\
        \frac{d}{dt} \left( \frac{\partial \mathcal{T}}{\partial \dot{q_2}} \right) - \frac{\partial \mathcal{T}}{\partial q_2} + \frac{\partial \mathcal{D}}{\partial \dot{q_2}} + \frac{\partial \mathcal{U}}{\partial q_2} = \mathcal{Q}
    \end{cases}
\end{equation}

By substituting the energetic terms obtained in Equations \ref{eq:kinetic_energy}, \ref{eq:dissipated_energy}, \ref{eq:potential_energy}, \ref{eq:generalized_input} into the set of equations above, we obtain the following equations of motion:

\begin{equation}
    \begin{cases}
        m \ddot{z} - \frac{1}{2} \frac{\partial L_1}{\partial z} \dot{q_1}^2 - \frac{1}{2} \frac{\partial L_2}{\partial z} \dot{q_2}^2 + \frac{1}{2} C_d A \rho \dot{z} |\dot{z}| - m g = 0                                                                                                                                                                                                            \\
        \frac{1}{2} \left( \frac{\partial^2 L_1}{\partial \dot{q_1} \partial z} \dot{z} + \frac{\partial^2 L_1}{\partial \dot{q_1}^2} \ddot{q_1} \right) \dot{q_1}^2 + \frac{\partial L_1}{\partial \dot{q_1}} \dot{q_1} \ddot{q_1} + \left( \frac{\partial L_1}{\partial z} \dot{z} + \frac{\partial L_1}{\partial \dot{q_1}} \ddot{q_1} \right) \dot{q_1} + L_1 \ddot{q_1} + R_1 \dot{q_1} - V_1 = 0 \\
        \frac{1}{2} \left( \frac{\partial^2 L_2}{\partial \dot{q_2} \partial z} \dot{z} + \frac{\partial^2 L_2}{\partial \dot{q_2}^2} \ddot{q_2} \right) \dot{q_2}^2 + \frac{\partial L_2}{\partial \dot{q_2}} \dot{q_2} \ddot{q_2} + \left( \frac{\partial L_2}{\partial z} \dot{z} + \frac{\partial L_2}{\partial \dot{q_2}} \ddot{q_2} \right) \dot{q_2} + L_2 \ddot{q_2} + R_2 \dot{q_2} - V_2 = 0 \\
    \end{cases}
\end{equation}

For convenience, we can replace time derivatives of charges leveraging the definition of current as the time derivative of the charge.
Moreover, we can group the terms in the equations above so to move derivatives with respect to the generalized coordinates on the left-hand side of the equations.
Finally, we also transform the second order differential equations into first order differential equations by introducing a fourth equation and considering the ball velocity $v$ as a state variable.
By doing so, we obtain the following set of equations:

\begin{equation}
    \begin{cases}
        \dot{z} = v                                                                                                                                                                                                                                                                                       \\
        \dot{v} = m^{-1} \left(\frac{1}{2} \frac{\partial L_1}{\partial z} I_1^2 + \frac{1}{2} \frac{\partial L_2}{\partial z} I_2^2 - \frac{1}{2} C_d A \rho \dot{z} |\dot{z}| + m g  \right)                                                                                                            \\

        \dot{I_1} = \left( \frac{1}{2} \frac{\partial^2 L_1}{\partial I_1^2} I_1^2 + 2\frac{\partial L_1}{\partial I_1} I_1 + L_1 \right)^{-1} \left( -\frac{1}{2} \frac{\partial^2 L_1}{\partial I_1 \partial z} \dot{z} I_1^2 - \frac{\partial L_1}{\partial z} \dot{z} I_1 - R_1 I_1 + V_1 \right) = 0 \\
        \dot{I_2} = \left( \frac{1}{2} \frac{\partial^2 L_2}{\partial I_2^2} I_2^2 + 2\frac{\partial L_2}{\partial I_2} I_2 + L_2 \right)^{-1} \left( -\frac{1}{2} \frac{\partial^2 L_2}{\partial I_2 \partial z} \dot{z} I_2^2 - \frac{\partial L_2}{\partial z} \dot{z} I_2 - R_2 I_2 + V_2 \right) = 0
    \end{cases}
    \label{eq:equations_of_motion_full}
\end{equation}

The set of equations above represents the complete mathematical model of the \acrshort{mls} system.
% One can notice that the equations are both nonlinear and coupled, making the system hard to analyze and control.



\subsubsection{Model reduction}
\label{subsubsec:model_reduction}

In order to simplify the model and make it more suitable for control purposes, we can make some assumptions that allow us to reduce the complexity of the system without losing its main dynamics.

Based also on the experimental data collected during the parameters' identification phase (Section \ref{sec:identification}), we can state that the sensitivity of the inductance terms to the current flowing through the coils is negligible around the operating point.
Moreover, also the velocity of the ball will always be small, and so every term that is linearly dependent on it can be neglected.
Based on these assumptions, we can impose the following conditions to the system:

\begin{equation}
    \begin{cases}
        \frac{\partial L_*}{\partial I_*}     & \approx 0 \\
        \frac{\partial^2 L_*}{\partial I_*^2} & \approx 0 \\
        \dot{z}                               & \approx 0
    \end{cases}
    \label{eq:model_reduction_conditions}
\end{equation}

Equations of motion \ref{eq:equations_of_motion_full} consequently simplify to:

\begin{equation}
    \begin{cases}
        \dot{z} = v                                                                                                                                 \\
        \dot{v} = m^{-1} \left(\frac{1}{2} \frac{\partial L_1}{\partial z} I_1^2 + \frac{1}{2} \frac{\partial L_2}{\partial z} I_2^2 + m g  \right) \\
        \dot{I_1} = L_1^{-1} \left(- R_1 I_1 + V_1 \right)                                                                                          \\
        \dot{I_2} = L_2^{-1} \left(- R_2 I_2 + V_2 \right)
    \end{cases}
    \label{eq:reduced_equations_of_motion}
\end{equation}



\subsubsection{Control input correction}
\label{subsubsec:control_input_correction}

A final important remark has to be made about the input given to the system.

So far, in all the equations above, we have considered the input to the system as the voltage applied to the coils.
However, the actual input to the system is the duty cycle of the PWM\footnote{Pulse-Width Modulation} signal that drives the coils.

Ideally, the relationship between the duty cycle and the voltage applied to the coils is a direct proportionality, but in practice, a linear approximation must be made.
One can easily see that by simply connecting the power supply to the coils, a minimum voltage will be applied and a certain amount of current will flow through.
In the following, we will refer to this current and voltage as $I_{*min}$ and $V_{*min}$ respectively.
In the context of control purposes, these zones where it's not possible to have a direct control over the applied voltage are called \textit{black zones}.

Under these considerations, one can rewrite the voltages applied to the coils $V_{*}$ as a function of the duty cycle $U_{*}$ as a piece wise linear function:

\begin{equation}
    V_{*} =
    \begin{cases}
        k_{*} U_{*} + c_{*} & \text{if } U_{*} \geq U_{*min} \\
        V_{*min}            & \text{if } U_{*} < U_{*min}
    \end{cases}
    \label{eq:voltage_duty_cycle_relation}
\end{equation}

Where $k_{*}$ and $c_{*}$ are the slope and the intercept of the linear relation, respectively, and $U_{*min}$ is the minimum duty cycle that allows to move outside the black zone and have a direct control over the voltage applied to the coils.

By substituting the voltages in the set of Equations \ref{eq:reduced_equations_of_motion}, we obtain the final set of equations that govern the system:

\begin{equation}
    \begin{cases}
        \dot{z} = v                                                                                                                                 \\
        \dot{v} = m^{-1} \left(\frac{1}{2} \frac{\partial L_1}{\partial z} I_1^2 + \frac{1}{2} \frac{\partial L_2}{\partial z} I_2^2 + m g  \right) \\
        \dot{I_1} = L_1^{-1} \left(- R_1 I_1 + (k_1 U_1 + c_1) \right)                                                                              \\
        \dot{I_2} = L_2^{-1} \left(- R_2 I_2 + (k_2 U_2 + c_2) \right)
    \end{cases}
    \label{eq:reduced_equations_of_motion_final}
\end{equation}

\subsection{Model Linearization}
\label{subsec:model_linearization}

The model derived in the previous subsections (Equations \ref{eq:reduced_equations_of_motion_final}) is highly non-linear.
In order to be able to apply linear control techniques, it is necessary to linearize the model around a given operating point.



\subsubsection{Operating point computation}
\label{subsubsec:operating_point_computation}

The operating point is the set of values of the state and input around which the linearization is performed.

Given the set of Equations \ref{eq:reduced_equations_of_motion_final}, the operating point can be computed by setting the time derivatives to zero, set at least 2 of the state variables or input variables to constant values and solve the remaining equations.
Based on their physical meaning, it's reasonable to set the position of the ball $z$ and the current in the lower electromagnet $I_2$.
By doing so, all the other state and input variables can be computed by solving the following set of equations:

\begin{equation}
    \mathbf{x}_{op} =
    \begin{bmatrix}
        z_{op}  \\
        v_{op}  \\
        I_{1op} \\
        I_{2op}
    \end{bmatrix}
    =
    \begin{cases}
        z^*                                                                                                                          \\
        0                                                                                                                            \\
        \sqrt{ -(2m g + \frac{\partial L_2}{\partial z} \big|_{z_{op}} I_{2op}^2) / \frac{\partial L_1}{\partial z} \big|_{z_{op}} } \\
        I_{2}^*
    \end{cases}
    \label{eq:operating_point_states}
\end{equation}

\begin{equation}
    \mathbf{u}_{op} =
    \begin{bmatrix}
        U_{1op} \\
        U_{2op}
    \end{bmatrix}
    =
    \begin{cases}
        \max{\left[0, R_{10} \left( I_{1op} - I_{1min} \right) / k_1 \right]} \\
        \max{\left[0, R_{20} \left( I_{2op} - I_{2min} \right) / k_2 \right]}
    \end{cases}
    \label{eq:operating_point_inputs}
\end{equation}

Where $z^*$ is the desired position of the ball and $I_{2}^*$ is the desired current in the lower electromagnet.
As we can see, once those values are set, all the other states and inputs can be computed uniquely.



\subsubsection{Linearized model derivation}
\label{subsubsec:linearized_model_derivation}

Based on the operating point computed in the previous subsection, the linearized model can be obtained by performing a Taylor expansion around the operating point up to the first order terms of Equations \ref{eq:reduced_equations_of_motion_final}.

Before performing the linearization, we briefly recall the general form of a Taylor expansion of a function $f(\mathbf{x})$ around a point $\mathbf{x}_{op}$:

\begin{equation}
    f(\mathbf{x}) \approx f(\mathbf{x}_{op}) + \nabla f(\mathbf{x}_{op}) \cdot (\mathbf{x} - \mathbf{x}_{op})
\end{equation}

Where $\nabla f(\mathbf{x}_{op})$ is the gradient of $f(\mathbf{x})$ evaluated at $\mathbf{x}_{op}$.

By applying the Taylor expansion to the non-linear model, the linearized model can be obtained as:

\begin{equation}
    \mathbf{f}(\mathbf{x}) - \mathbf{f}(\mathbf{x}_{op})\approx \frac{\partial \mathbf{f}}{\partial \mathbf{x}} \Bigg|_{\mathbf{x}_{op}} \cdot (\mathbf{x} - \mathbf{x}_{op})
\end{equation}

Considering now the set of Equations \ref{eq:reduced_equations_of_motion_final}, the linearized model can be obtained as:

\small
\begin{equation}
    \begin{cases}
        \dot{z} - \dot{z_{op}}    & \approx 1 (v - v_{op})                  \\
        \dot{v} - \dot{v_{op}}    & \approx m^{-1} \left(
        \frac{1}{2} \frac{\partial^2 L_1}{\partial z^2} \Bigg|_{\mathbf{x}_{op}} (z - z_{op}) I_{1op}^2 +
        \frac{1}{2} \frac{\partial^2 L_2}{\partial z^2} \Bigg|_{\mathbf{x}_{op}} (z - z_{op}) I_{2op}^2 +
        \frac{\partial L_1}{\partial z} \Bigg|_{\mathbf{x}_{op}} I_{1op} (I_1 - I_{1op}) +
        \frac{\partial L_2}{\partial z} \Bigg|_{\mathbf{x}_{op}} I_{2op} (I_2 - I_{2op})
        \right)                                                             \\
        \dot{I_1} - \dot{I_{1op}} & \approx
        \left(- L_1^{-2} \frac{\partial L_1}{\partial z} \left(- R_1 I_1 + k_1 U_1 + c_1 \right) \right) \Bigg|_{\mathbf{x}_{op}} (z - z_{op}) +
        \left(- L_1^{-1} R_1 \right) \Bigg|_{\mathbf{x}_{op}} (I_1 - I_{1op}) +
        \left(L_1^{-1} k_1 \right) \Bigg|_{\mathbf{x}_{op}} (U_1 - U_{1op}) \\
        \dot{I_2} - \dot{I_{2op}} & \approx
        \left(- L_2^{-2} \frac{\partial L_2}{\partial z} \left(- R_2 I_2 + k_2 U_2 + c_2 \right) \right) \Bigg|_{\mathbf{x}_{op}} (z - z_{op}) +
        \left(- L_2^{-1} R_2 \right) \Bigg|_{\mathbf{x}_{op}} (I_2 - I_{2op}) +
        \left(L_2^{-1} k_2 \right) \Bigg|_{\mathbf{x}_{op}} (U_2 - U_{2op})
    \end{cases}
    \label{eq:linearized_model}
\end{equation}
\normalsize

Notice that also during the linearization process, the model has been simplified by reapplying the assumptions made in the set of Equations \ref{eq:model_reduction_conditions}.
\subsection{State Space Representation}
\label{subsec:state_space_representation}

In the optics of control theory, it is useful to represent the system in the state space form.
The state space representation is a mathematical model of a physical system as a set of input, output and state variables related by first-order differential equations.
The state space representation is particularly useful for linear systems, as it allows to easily apply linear control techniques.

A generic nonlinear system can be represented in the state space form as:

\begin{equation}
    \begin{aligned}
        \dot{\mathbf{x}} & = f(\mathbf{x}, \mathbf{u}) \\
        \mathbf{y}       & = g(\mathbf{x}, \mathbf{u})
    \end{aligned}
\end{equation}

Where $\mathbf{x}$ is the state vector and $\mathbf{u}$ is the input vector.

Similarly to what has been done in the previous subsection, we can perform a linearization of the system around an operating point to obtain the linearized state space representation in the form of:

\begin{equation}
    \begin{aligned}
        \dot{\delta\mathbf{x}} & \approx A \delta\mathbf{x} + B \delta\mathbf{u} \\
        \delta\mathbf{y}       & \approx C \delta\mathbf{x} + D \delta\mathbf{u}
    \end{aligned}
\end{equation}

Where $\delta\mathbf{x}$ and $\delta\mathbf{u}$ are the deviations of the state and input vectors from the operating point, respectively.
While the matrices $A$, $B$, $C$ and $D$ are the Jacobian matrices with respect to the state and input vectors evaluated at the operating point.

\paragraph{MagLev System State Space Representation}

Given the linearized model derived in the previous subsection, we can define the state vector $\mathbf{x}$ and the input vector $\mathbf{u}$ as:

\begin{equation}
    \mathbf{x} = \begin{bmatrix}
        z   \\
        v   \\
        I_1 \\
        I_2
    \end{bmatrix}
    \quad
    \mathbf{u} = \begin{bmatrix}
        U_1 \\
        U_2
    \end{bmatrix}
\end{equation}

Once the state and input vectors have been defined, the linearized state space representation can be obtained by leveraging the linearized model derived in the previous subsection.
The matrices $A$, $B$, $C$ and $D$ are then defined as:

\begin{equation}
    \begin{aligned}
        A & = \frac{\partial f}{\partial \mathbf{x}} \Bigg|_{(\mathbf{x}_{op}, \mathbf{u}_{op})}
        = \begin{bmatrix}
              0      & 1 & 0      & 0      \\
              a_{21} & 0 & a_{23} & a_{24} \\
              a_{31} & 0 & a_{33} & 0      \\
              a_{41} & 0 & 0      & a_{44}
          \end{bmatrix}                                                           \\
        B & = \frac{\partial f}{\partial \mathbf{u}} \Bigg|_{(\mathbf{x}_{op}, \mathbf{u}_{op})}
        = \begin{bmatrix}
              0      & 0      \\
              0      & 0      \\
              b_{31} & 0      \\
              0      & b_{42}
          \end{bmatrix}                                                                        \\
        C & = \frac{\partial g}{\partial \mathbf{x}} \Bigg|_{(\mathbf{x}_{op}, \mathbf{u}_{op})}
        = \begin{bmatrix}
              1 & 0 & 0 & 0
          \end{bmatrix}                                                                         \\
        D & = \frac{\partial g}{\partial \mathbf{u}} \Bigg|_{(\mathbf{x}_{op}, \mathbf{u}_{op})}
        = \begin{bmatrix}
              0 & 0
          \end{bmatrix}
    \end{aligned}
\end{equation}

By leveraging the linearization of the model derived in the previous subsection and reported in Equation \ref{eq:linearized_model}, the elements of the matrices $A$, $B$, $C$ and $D$ can be computed as:

\begin{equation}
    \begin{aligned}
        a_{21} & = \frac{1}{m} \left( \frac{1}{2} \frac{\partial^2 L_1}{\partial z^2} I_1^2 + \frac{1}{2} \frac{\partial^2 L_2}{\partial z^2} I_{2}^2 \right) \bigg|_{(\mathbf{x}_{op}, \mathbf{u}_{op})} \\
        a_{23} & = \frac{1}{m} \left( \frac{\partial L_1}{\partial z} I_1\right) \bigg|_{(\mathbf{x}_{op}, \mathbf{u}_{op})}                                                                              \\
        a_{24} & = \frac{1}{m} \left( \frac{\partial L_2}{\partial z} I_2\right) \bigg|_{(\mathbf{x}_{op}, \mathbf{u}_{op})}                                                                              \\
        a_{31} & = \left(- L_1^{-2} \frac{\partial L_1}{\partial z} \left(- R_1 I_1 + k_1 U_1 + c_1 \right) \right) \bigg|_{(\mathbf{x}_{op}, \mathbf{u}_{op})}                                           \\
        a_{33} & = \left(L_1^{-1} (-R_1) \right) \big|_{(\mathbf{x}_{op}, \mathbf{u}_{op})}                                                                                                               \\
        a_{41} & = \left(- L_2^{-2} \frac{\partial L_2}{\partial z} \left(- R_2 I_2 + k_2 U_2 + c_2 \right) \right) \bigg|_{(\mathbf{x}_{op}, \mathbf{u}_{op})}                                           \\
        a_{44} & = \left(L_2^{-1} (-R_2) \right) \big|_{(\mathbf{x}_{op}, \mathbf{u}_{op})}                                                                                                               \\
        b_{31} & = \left(L_1^{-1} k_1 \right) \big|_{(\mathbf{x}_{op}, \mathbf{u}_{op})}                                                                                                                  \\
        b_{42} & = \left(L_2^{-1} k_2 \right) \big|_{(\mathbf{x}_{op}, \mathbf{u}_{op})}
    \end{aligned}
\end{equation}



\subsection{Single Coil Configuration}
\label{subsec:single_coil_configuration}

In this section, we will present the model of the \acrshort{mls} system when only the upper coil is used for control purposes.
This configuration is the one that will be used in the rest of the document and the laboratory activities.
The choice has been taken in order to deal with a simpler SISO system, which is easier to control and analyze.

In the following, starting from Equation \ref{eq:reduced_equations_of_motion_final}, we will at derive the reduced model, linearize it and represent it in state-space form.

\paragraph{Reduced Equations of Motion}

At first, if we consider energizing only the upper coil, we can simply remove the terms related to the lower coil from the equations of motion.
Based on Equation \ref{eq:reduced_equations_of_motion_final}, we can write the following equations:

\begin{equation}
    \begin{cases}
        \dot{z} = v                                                                             \\
        \dot{v} = m^{-1} \left(\frac{1}{2} \frac{\partial L_1}{\partial z} I_1^2 + m g  \right) \\
        \dot{I_1} = L_1^{-1} \left(- R_1 I_1 + (k_1 U_1 + c_1) \right)
    \end{cases}
    \label{eq:equations_of_motion_single_coil}
\end{equation}

\paragraph{Linearization}

As already discussed in Section \ref{subsec:model_linearization}, we can linearize via Taylor expansion the equations of motion around one of its operating points.
For the case of the single coil configuration, Equations \ref{eq:operating_point_states} and \ref{eq:operating_point_inputs} reduce to:

\begin{equation}
    \mathbf{x}_{op} =
    \begin{bmatrix}
        z_{op} \\
        v_{op} \\
        I_{1op}
    \end{bmatrix}
    =
    \begin{cases}
        z^* \\
        0   \\
        \sqrt{ -(2m g) / \frac{\partial L_1}{\partial z} \big|_{z_{op}} }
    \end{cases}
    \label{eq:operating_point_states_single_coil}
\end{equation}

\begin{equation}
    \mathbf{u}_{op} =
    \begin{bmatrix}
        U_{1op}
    \end{bmatrix}
    =
    \begin{cases}
        \max{\left[0, R_{10} \left( I_{1op} - I_{1min} \right) / k_1 \right]}
    \end{cases}
    \label{eq:operating_point_inputs_single_coil}
\end{equation}

By performing the Taylor expansion of Equations \ref{eq:equations_of_motion_single_coil} around the operating point, we obtain the following linearized model:

\begin{equation}
    \begin{cases}
        \dot{z} - \dot{z_{op}}    & \approx 1 (v - v_{op}) \\
        \dot{v} - \dot{v_{op}}    & \approx m^{-1} \left(
        \frac{1}{2} \frac{\partial^2 L_1}{\partial z^2} \Bigg|_{\mathbf{x}_{op}} (z - z_{op}) I_{1op}^2 +
        \frac{\partial L_1}{\partial z} \Bigg|_{\mathbf{x}_{op}} I_{1op} (I_1 - I_{1op})
        \right)                                            \\
        \dot{I_1} - \dot{I_{1op}} & \approx
        \left(- L_1^{-2} \frac{\partial L_1}{\partial z} \left(- R_1 I_1 + k_1 U_1 + c_1 \right) \right) \Bigg|_{\mathbf{x}_{op}} (z - z_{op}) +
        \left(- L_1^{-1} R_1 \right) \Bigg|_{\mathbf{x}_{op}} (I_1 - I_{1op}) +
        \left(L_1^{-1} k_1 \right) \Bigg|_{\mathbf{x}_{op}} (U_1 - U_{1op})
    \end{cases}
    \label{eq:linearized_model_single_coil}
\end{equation}

\paragraph{State-Space Representation}

Finally, we can represent the linearized model in state-space form.
Given the reduction of the system to a SISO one, we need to redefine the state vector $\mathbf{x}$ and the input vector $\mathbf{u}$ as follows:

\begin{equation}
    \mathbf{x} = \begin{bmatrix}
        z \\
        v \\
        I_1
    \end{bmatrix}
    \quad
    \mathbf{u} = \begin{bmatrix}
        U_1
    \end{bmatrix}
\end{equation}


Once the state and input vectors have been defined, the linearized state-space representation can be obtained by leveraging the linearized model derived previously.
The matrices $A$, $B$, $C$ and $D$ are then defined as:

\begin{equation}
    \begin{aligned}
        A & = \frac{\partial f}{\partial \mathbf{x}} \Bigg|_{(\mathbf{x}_{op}, \mathbf{u}_{op})}
        = \begin{bmatrix}
              0      & 1 & 0      \\
              a_{21} & 0 & a_{23} \\
              a_{31} & 0 & a_{33} \\
          \end{bmatrix}                                                                    \\
        B & = \frac{\partial f}{\partial \mathbf{u}} \Bigg|_{(\mathbf{x}_{op}, \mathbf{u}_{op})}
        = \begin{bmatrix}
              0 \\
              0 \\
              b_{31}
          \end{bmatrix}                                                                         \\
        C & = \frac{\partial g}{\partial \mathbf{x}} \Bigg|_{(\mathbf{x}_{op}, \mathbf{u}_{op})}
        = \begin{bmatrix}
              1 & 0 & 0
          \end{bmatrix}                                                                         \\
        D & = \frac{\partial g}{\partial \mathbf{u}} \Bigg|_{(\mathbf{x}_{op}, \mathbf{u}_{op})}
        = \begin{bmatrix}
              0
          \end{bmatrix}
    \end{aligned}
\end{equation}

Given the unboundedness between the two coils currents, the elements of the matrices $A$, $B$, $C$ and $D$ remain exactly as already computed in Section \ref{eq:state_space_matrices}.