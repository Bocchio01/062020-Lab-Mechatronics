\subsection{Sensors characterization}
\label{subsec:sensors_characterization}

\hl{To be rewritten in a better form.}

We need to create the mapping between the position of the ball and the output voltage of the infrared sensor.
To do so, we simply create an array of points where we measure the position of the ball using a caliper and the output voltage of the infrared sensor using the data acquisition system included in the \texttt{Inteco} control unit.

The obtained data is shown in Figure \ref{fig:position_to_voltage}.

\begin{figure}[H]
    \centering
    % \includegraphics[width=0.6\textwidth]{img/MATLAB/identification/position_to_voltage.pdf}
    \caption{Position to voltage identification}
    \label{fig:position_to_voltage}
\end{figure}

As we can see, the relation between the position of the ball and the output voltage of the infrared sensor can be approximated as linear outside the limits of the sensor.

From the data obtained, one can clearly see that for distances of the ball from the upper coil grater than $\approx 20 [mm]$, the sensor reaches its saturation limit and the output voltage is constant.
Based on this observation, we can also impose that during simulations, the maximum distance of the ball from the upper coil is $20 [mm]$.

As a second step, we also need to study the variance of all the sensors used internally in the control unit.
We now suppose for both the infrared sensor for position and the current sensor that their model is Gaussian.
Based on this assumption, one can compute