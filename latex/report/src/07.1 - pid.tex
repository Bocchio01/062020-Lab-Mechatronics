\subsection{PID Controllers}
\label{subsec:pid_controllers}

The Proportional-Integral-Derivative (PID) controller is a simple controller that uses the error signal, its history and derivative to compute the control signal.
It is a widely used controller in industry due to its simplicity and effectiveness in many applications.

The PID controller is defined by the following equation:

\begin{equation}
    u(t) = K_p e(t) + K_i \int_{0}^{t} e(\tau)dt + K_d \frac{de(t)}{dt} = K_p \left(e(t) + \frac{1}{T_i} \int_{0}^{t} e(\tau)dt + T_d \frac{de(t)}{dt}\right)
\end{equation}

Where $K_p$, $K_i$ and $K_d$ are the proportional, integral and derivative gains, respectively.
$T_i$ and $T_d$ instead are the integral and derivative time constants, respectively.


\subsubsection{PID classic}
\label{subsubsec:pid_classic}

In its simplest form, the PID is a linear controller whose three gains are tuned based on the linearization of the system.



\subsubsection{PID with Anti-Windup correction}
\label{subsubsec:pid_anti_windup}



\subsubsection{PID with gain scheduling}
\label{subsubsec:pid_gain_scheduling}
