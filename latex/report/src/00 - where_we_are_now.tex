\section{Where we are now}
\label{sec:where_we_are_now}

For now, we are ignoring the second coil and have a (possibly wrong) modelling of both the electromagnetic and mechanical subsystems of the \acrshort{mls}.

Keep always in mind that:
\begin{itemize}
    \item What we want to control is the position of the ball, $z$.
    \item The input of the overall system, will always be the target postion $z_{\text{ref}}$.
    \item The output of the overall system, will always be the actual position $z$.
    \item Our control logic, should then received the some kind of position error, $e = z_{\text{ref}} - z$ and output the voltage to apply to the coil.
\end{itemize}

\begin{equation}
    \begin{cases}
        \dot{i} = - \frac{R}{L} \cdot i + \frac{\Delta V}{L} \\
        \dot{z} = v                                          \\
        \dot{v} = g - \frac{F_{em}}{m}
    \end{cases}
\end{equation}

From other online resources and thesis from others, we have seen that the magnetic force have been modelled as:

\begin{equation}
    F_{em} = k_v \frac{i^2}{z^2}
\end{equation}

\textbf{However}, this simple model doesn't take into account any inductive perturbation that an object immersed in a magnetic field produces on the field itself.
For example, the manual on the MLS2EM, has some formulas (never seen before) that somehow take into account this effect.

\paragraph{State space representation}

\begin{equation}
    \begin{cases}
        \dot{x} = A x + B u \\
        y = C x
    \end{cases}
\end{equation}

\begin{equation}
    x = \begin{bmatrix}
        i \\
        z \\
        v
    \end{bmatrix}
    \quad
    u = \Delta V
    \quad
    y = z
\end{equation}

\begin{equation}
    A = \begin{bmatrix}
        -\frac{R}{L}  & 0                & 0           \\
        0             & 0                & 1           \\
        \text{Taylor} & \text{expansion} & \text{here}
    \end{bmatrix}
    \quad
    B = \begin{bmatrix}
        \frac{1}{L} \\
        0           \\
        0
    \end{bmatrix}
    \quad
    C = \begin{bmatrix}
        0 & 1 & 0
    \end{bmatrix}
\end{equation}