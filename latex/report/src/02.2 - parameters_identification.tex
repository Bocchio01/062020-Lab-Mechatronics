\subsection{Parameters identification}
\label{subsec:parameters_identification}

In order to control the system, we need to identify its parameters.

The unknown parameters of the system are depicted in table \ref{tab:parameters}.

\begin{table}[H]
    \centering
    \begin{tabular}{|c|l|c|}
        \hline
        Parameter         & Description                                        & Unit          \\
        \hline
        $R_{1, 2}$        & Resistance of the coil                             & $\Omega$      \\
        $L_{1, 2}$        & Inductance of the coil                             & $H$           \\
        $\frac{d L}{d z}$ & Sensitivity of the inductance to the ball position & $\frac{H}{m}$ \\
        \hline
    \end{tabular}
\end{table}

In order to indentify the parameters of the system, we can use a \texttt{PICO} oscillometer to measure the current going trough the coil and the voltage applied to it.
By doing so, we can fully identify the missig parameters of the system, such as the resistance of the coil $R [\Omega]$, inductance of the coil $L [H]$ and the sensitivity of the inductance to the ball position $\frac{d L}{d z} [\frac{H}{m}]$.







Based on the MLS2EM datasheet/manual (pages 6-18), the procedure to identify the parameters of the system should be as follows.

Launch \texttt{mls2em-usb2-main} script and open tools/identification window and then:

\begin{enumerate}
    \item Optical sensor calibration: it's the curve the mesured voltage from the optical sensor to the effective height of the ball. It's used internally by the \texttt{MLS-SIM-BLOCK}. Output: mls2em-usb2-Sensor.mat->SensorData.[Distance-mm, Sensor-V], $z = f(\Delta V)$.
    \item Static characterization (coil): it's the curve of the current in the coil due to a given applied voltage. It's assumed to be linear $i = f(\Delta V) = k_i \Delta V  + c_i$, but with a deadzone at the beginning $i_{MIN} = f(\Delta V_{MIN})$. Output: $k_i [\frac{A}{V}]$, $c_i [A]$, $i_{MIN} [A]$, $V_{MIN} [V]$.
    \item Minimal effort to move the ball: from this we can have a relation between the current in the coil and the force applied to the ball.
    \item Dynamic characterization (coil): as we know, obkect moving in a magnetic field generate distorsion in the field itself/changes in the coil's current. All these effects are taken into account by $K_i$ and $f_i$. Basically, the plot on page 18, shows the solution for a classical RL circuit $i(t) = i_0 + i_{\infty} (1 - e^{-\frac{t}{\tau}})$, derived from the differential equation $\dot{i} = \frac{V - R i}{L}$. It's unclear what $K_i$ and $f_i$ really are.
\end{enumerate}


So, brief recap of the parameters that the manual uses in its mathematical model:

\begin{table}[H]
    \centering
    \begin{tabular}{|c|l|c|c|}
        \hline
        Parameter  & Description                           & Unit            & Notes           \\
        \hline
        $m$        & Ball mass                             & $kg$            &                 \\
        $g$        & Gravity acceleration                  & $\frac{m}{s^2}$ &                 \\
        $F_{emP1}$ & Magnetic modelling related            & $H$             & To be clarified \\
        $F_{emP2}$ & Magnetic modelling related            & $m$             & To be clarified \\
        $f_{iP1}$  & Inductance modelling related          & $m \cdot s$     & To be clarified \\
        $f_{iP2}$  & Inductance modelling related          & $m$             & To be clarified \\
        $k_i$      & Basically it's the conductance $1/R$  & $\frac{A}{V}$   &                 \\
        $c_i$      & Coil offset                           & $A$             &                 \\
        $i_{MIN}$  & Minimum current (deadzone of control) & $A$             &                 \\
        $V_{MIN}$  & Minimum voltage (deadzone of control) & $V$             &                 \\
        $K_i$      & Current rise modelling in coil        &                 &                 \\
        $f_i$      & Current rise modelling in coil        &                 &                 \\
        \hline
    \end{tabular}
    \caption{Parameters of the MLS2EM system}
    \label{tab:parameters}
\end{table}

\begin{table}[H]
    \centering
    \begin{tabular}{|c|l|c|}
        \hline
        Function   & Description                                                                             & Unit \\
        \hline
        $F_g$      & Gravity force acting on the ball                                                        & $N$  \\
        $F_{em1}$  & Electromagnetic force from upper coil                                                   & $N$  \\
        $F_{em2}$  & Electromagnetic force from lower coil                                                   & $N$  \\
        $f_i(x_1)$ & Function to model the variation of the inductance in the coils due to the ball position & $s$  \\
        \hline
    \end{tabular}
    \caption{Functions of the MLS2EM system}
    \label{tab:functions}
\end{table}