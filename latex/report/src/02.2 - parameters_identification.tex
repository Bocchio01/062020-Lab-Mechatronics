\subsection{Parameters identification}
\label{subsec:parameters_identification}

In order to control the system, we need to identify its parameters.

To do so, different tests type will be performed, such as:

\begin{enumerate}
    \item \textbf{Direct Measurement:} The first step is to measure all the quantity of the system that can be directly measured or retrived from well established literature.
    \item \textbf{RL Circuit Identification:} The first step is to identify the parameters of the RL circuit. This will be done by applying a step input to the system and measuring the current;
    \item \textbf{Characterization of ball influence:} The second step is to identify how the presence of the ball affects the coils and the magnetic field. This will be done by measuring the inductance of the coils with the ball positioned at different heights.
\end{enumerate}

All the above tests that involve any kind of measurement from the system will be performed leveraging the data acquisition system included in the \texttt{Inteco} control unit.

Based on the mathematical model developed in Section \ref{subsec:model_derivation} and the set of Equations \ref{eq:equations_of_motion} and Equations \ref{eq:equations_of_motion_simplified}, the following parameters need to be identified:

\begin{table}[H]

    \centering
    \begin{tabular}{|c|l|c|c|}
        \hline
        \textbf{Name}     & \textbf{Description}               & \textbf{Unit} & \textbf{Test \#} \\
        \hline
        $m$               & Mass of the ball                   & $kg$          & 1                \\
        $r$               & Radius of the ball                 & $m$           & 1                \\
        $H$               & Distance between the two coils     & $m$           & 1                \\
        $C_d$             & Drag coefficient                   & $N \cdot s/m$ & 1                \\
        $\rho$            & Density of the air                 & $kg/m^3$      & 1                \\
        $L_{*0}$          & Static inductance of the coils     & $H$           & 2                \\
        $L_{*z}$ \& $a_*$ & Dynamic inductance characteristics & $H$ \& $1/m$  & 3                \\
        $R_{*0}$          & Static resistance of the coils     & $\Omega$      & 2                \\
        $f_{IP1}$         & Empirical parameters               & $m\cdot s$    & /                \\
        $f_{IP2}$         & Empirical parameters               & $m$           & /                \\
        $I_{*min}$        & Minimum current                    & $A$           & 2                \\
        $I_{*max}$        & Maximum current                    & $A$           & 2                \\
        $V_{*min}$        & Minimum voltage                    & $V$           & 2                \\
        $V_{*max}$        & Maximum voltage                    & $V$           & 2                \\
        \hline
    \end{tabular}

    \caption{Parameters to be identified}
    \label{tab:parameters_to_be_identified}

\end{table}

Notice that column \textbf{Test \#} in Table \ref{tab:parameters_to_be_identified} refers to the test ID in the list above.

\paragraph{Direct Measurement}

The mass of the ball and its radius can be directly measured using a scale and a caliper, respectively.
Also the distance between the two coils can be measured using a caliper.

The drag coefficient can be obtained from well established literature.
In the following, we will use the drag coefficient of a sphere in a fluid and we will assume it to be constant for all the range of velocities we are going to work with.
This, for relatively low velocities, can be considered a good approximation.

The density of the air can be obtained from well established literature.

By doing so, we will have the following values:

\begin{table}[H]

    \centering
    \begin{tabular}{|c|c|}
        \hline
        \textbf{Parameter} & \textbf{Value}        \\
        \hline
        $m$                & $0.06157 \, kg$       \\
        $r$                & $0.06125/2 \, m$      \\
        $H$                & $0.098 \, m$          \\
        $C_d$              & $0.45 \, N \cdot s/m$ \\
        $\rho$             & $1.21 \, kg/m^3$      \\
        \hline
    \end{tabular}

    \caption{Directly measured parameters}
    \label{tab:directly_measured_parameters}

\end{table}

\paragraph{RL Circuit Identification}

The RL circuit identification will be done by applying a step input to the system and measuring the current.
The current will be measured using the data acquisition system included in the \texttt{Inteco} control unit.

The identification will be done by fitting the measured current to the following model:

\begin{equation}
    i(t) = \frac{V_{*max}}{R_{*0}} \left(1 - e^{-\frac{R_{*0}}{L_{*0}}t}\right)
\end{equation}

Also the minimum and maximum current and voltage can be measured directly using the data acquisition system.

In Figure \ref{fig:RL_circuit_identification} we can see 10 transient responses of the current in the RL circuit (black dashed lines) as well as the fitted model (red solid line).

\begin{figure}[H]
    \centering
    % \includegraphics[width=0.8\textwidth]{img/RL_circuit_identification.png}
    \caption{RL circuit identification}
    \label{fig:RL_circuit_identification}
\end{figure}

By doing so, we will have the following values:

\begin{table}[H]

    \centering
    \begin{tabular}{|c|c|}
        \hline
        \textbf{Parameter} & \textbf{Value} \\
        \hline
        $L_{*0}$           & $? \, H$       \\
        $R_{*0}$           & $? \, \Omega$  \\
        $I_{*min}$         & $? \, A$       \\
        $I_{*max}$         & $? \, A$       \\
        $V_{*min}$         & $? \, V$       \\
        $V_{*max}$         & $? \, V$       \\
        \hline
    \end{tabular}

    \caption{RL circuit identification parameters}
    \label{tab:RL_circuit_identification_parameters}

\end{table}

\paragraph{Characterization of ball influence}

The characterization of the ball influence will be done by measuring the inductance of the coils with the ball positioned at different heights.

For convenience and time saving, we will adopt a different strategy that will allow us to measure directly the sensitivity of the inductance to the ball position.
To do so, we recall Equation \ref{eq:equations_of_motion} and in particular the equation relative to $\dot{v}$:

\begin{equation}
    \dot{v} = m^{-1} \left(\frac{1}{2} \frac{\partial L_1}{\partial z} I_1^2 + \frac{1}{2} \frac{\partial L_2}{\partial z} I_2^2 - \frac{1}{2} C_d A \rho \dot{z}^2 + m g  \right)
\end{equation}

If we consider the system at rest, we can simplify the equation as follows:

\begin{equation}
    0 = \frac{1}{2} \frac{\partial L_1}{\partial z} I_1^2 + \frac{1}{2} \frac{\partial L_2}{\partial z} I_2^2 + m g
\end{equation}

Supposing now that only the first coil is energized, we can further simplify the equation as follows:

\begin{equation}
    0 = \frac{1}{2} \frac{\partial L_1}{\partial z} I_1^2 + m g
\end{equation}

Which leads to:

\begin{equation}
    \frac{\partial L_1}{\partial z} = -2 \frac{m g}{I_1^2}
    \label{eq:sensitivity_of_inductance}
\end{equation}

This last equation basically tells us that in steady state conditions, when the ball is levitating (i.e. $\dot{z} = 0$ and not supported by any platform), the sensitivity of the inductance of the first coil has an analytical expression that can be directly measured by measuring the current in the first coil.

In order to follow this appoarch, a linearly increasing voltage has been applied to the first coil and the current corresponding to the levitation of the ball has been measured.

In Figure \ref{fig:levitation_current} we can see both the position of the ball and the current circulating in the first coil.
By indentifying the current at which the ball starts to levitate (i.e. the ball starts to move upwards), we can than use Equation \ref{eq:sensitivity_of_inductance} to identify the dynamic inductance characteristics.

\begin{figure}[H]
    \centering
    % \includegraphics[width=0.8\textwidth]{img/levitation_current.png}
    \caption{Levitation current}
    \label{fig:levitation_current}
\end{figure}

The test described above has been repeated for different heights of the ball, in order to fully characterize the dynamic inductance characteristics over the range of possible ball positions.
The results are shown in Table \ref{tab:dynamic_inductance_characteristics} and in Figure \ref{fig:dynamic_inductance_characteristics}.

\begin{table}[H]

    \centering
    \begin{tabular}{|c|c|c|c|}
        \hline
        \textbf{Position (z) [mm]} & \textbf{Current (I) [A]} & \textbf{Control Voltage (V) [V]} & \textbf{Sensitivity ($\partial L / \partial z$) [H/m]} \\
        \hline
        0.0                        & 0.0                      & 0.0                              & 0.0                                                    \\
        0.0                        & 0.0                      & 0.0                              & 0.0                                                    \\
        \hline
    \end{tabular}

    \caption{Dynamic inductance characteristics}
    \label{tab:dynamic_inductance_characteristics}

\end{table}

\begin{figure}[H]
    \centering
    % \includegraphics[width=0.8\textwidth]{img/dynamic_inductance_characteristics.png}
    \caption{Dynamic inductance characteristics}
    \label{fig:dynamic_inductance_characteristics}
\end{figure}
